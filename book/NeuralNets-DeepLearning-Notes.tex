% -*- TeX -*- -*- UK -*- -*- Soft -*-"

\documentclass[a4,nobib]{tufte-book} 
\usepackage[utf8]{inputenc}
\usepackage[T1]{fontenc}

\hypersetup{colorlinks}% uncomment this line if you prefer colored hyperlinks (e.g., for onscreen viewing)

%%
% Book metadata
\title{Neural Networks and Deep Learning}
\author{}
\publisher{}

%%
% For nicely typeset tabular material
\usepackage{booktabs}

%%
% For graphics / images
\usepackage{graphicx}
\setkeys{Gin}{width=\linewidth,totalheight=\textheight,keepaspectratio}
\graphicspath{{pic/}}

% The fancyvrb package lets us customize the formatting of verbatim
% environments.  We use a slightly smaller font.
\usepackage{fancyvrb}
\fvset{fontsize=\normalsize}

%%
% Prints argument within hanging parentheses (i.e., parentheses that take
% up no horizontal space).  Useful in tabular environments.
\newcommand{\hangp}[1]{\makebox[0pt][r]{(}#1\makebox[0pt][l]{)}}

%%
% Prints an asterisk that takes up no horizontal space.
% Useful in tabular environments.
\newcommand{\hangstar}{\makebox[0pt][l]{*}}

%%
% Prints a trailing space in a smart way.
\usepackage{xspace}


\newcommand{\TL}{Tufte-\LaTeX\xspace}

% Prints the month name (e.g., January) and the year (e.g., 2008)
\newcommand{\monthyear}{%
  \ifcase\month\or January\or February\or March\or April\or May\or June\or
  July\or August\or September\or October\or November\or
  December\fi\space\number\year
}


% Prints an epigraph and speaker in sans serif, all-caps type.
\newcommand{\openepigraph}[2]{%
  %\sffamily\fontsize{14}{16}\selectfont
  \begin{fullwidth}
  \sffamily\large
  \begin{doublespace}
  \noindent\allcaps{#1}\\% epigraph
  \noindent\allcaps{#2}% author
  \end{doublespace}
  \end{fullwidth}
}

% Inserts a blank page
\newcommand{\blankpage}{\newpage\hbox{}\thispagestyle{empty}\newpage}

\usepackage{units}

% Typesets the font size, leading, and measure in the form of 10/12x26 pc.
\newcommand{\measure}[3]{#1/#2$\times$\unit[#3]{pc}}

% Macros for typesetting the documentation
\newcommand{\hlred}[1]{\textcolor{Maroon}{#1}}% prints in red
\newcommand{\hangleft}[1]{\makebox[0pt][r]{#1}}
\newcommand{\hairsp}{\hspace{1pt}}% hair space
\newcommand{\hquad}{\hskip0.5em\relax}% half quad space
\newcommand{\TODO}{\textcolor{red}{\bf TODO!}\xspace}
\newcommand{\na}{\quad--}% used in tables for N/A cells
\providecommand{\XeLaTeX}{X\lower.5ex\hbox{\kern-0.15em\reflectbox{E}}\kern-0.1em\LaTeX}
\newcommand{\tXeLaTeX}{\XeLaTeX\index{XeLaTeX@\protect\XeLaTeX}}
% \index{\texttt{\textbackslash xyz}@\hangleft{\texttt{\textbackslash}}\texttt{xyz}}
\newcommand{\tuftebs}{\symbol{'134}}% a backslash in tt type in OT1/T1
\newcommand{\doccmdnoindex}[2][]{\texttt{\tuftebs#2}}% command name -- adds backslash automatically (and doesn't add cmd to the index)
\newcommand{\doccmddef}[2][]{%
  \hlred{\texttt{\tuftebs#2}}\label{cmd:#2}%
  \ifthenelse{\isempty{#1}}%
    {% add the command to the index
      \index{#2 command@\protect\hangleft{\texttt{\tuftebs}}\texttt{#2}}% command name
    }%
    {% add the command and package to the index
      \index{#2 command@\protect\hangleft{\texttt{\tuftebs}}\texttt{#2} (\texttt{#1} package)}% command name
      \index{#1 package@\texttt{#1} package}\index{packages!#1@\texttt{#1}}% package name
    }%
}% command name -- adds backslash automatically
\newcommand{\doccmd}[2][]{%
  \texttt{\tuftebs#2}%
  \ifthenelse{\isempty{#1}}%
    {% add the command to the index
      \index{#2 command@\protect\hangleft{\texttt{\tuftebs}}\texttt{#2}}% command name
    }%
    {% add the command and package to the index
      \index{#2 command@\protect\hangleft{\texttt{\tuftebs}}\texttt{#2} (\texttt{#1} package)}% command name
      \index{#1 package@\texttt{#1} package}\index{packages!#1@\texttt{#1}}% package name
    }%
}% command name -- adds backslash automatically
\newcommand{\docopt}[1]{\ensuremath{\langle}\textrm{\textit{#1}}\ensuremath{\rangle}}% optional command argument
\newcommand{\docarg}[1]{\textrm{\textit{#1}}}% (required) command argument
\newenvironment{docspec}{\begin{quotation}\ttfamily\parskip0pt\parindent0pt\ignorespaces}{\end{quotation}}% command specification environment
\newcommand{\docenv}[1]{\texttt{#1}\index{#1 environment@\texttt{#1} environment}\index{environments!#1@\texttt{#1}}}% environment name
\newcommand{\docenvdef}[1]{\hlred{\texttt{#1}}\label{env:#1}\index{#1 environment@\texttt{#1} environment}\index{environments!#1@\texttt{#1}}}% environment name
\newcommand{\docpkg}[1]{\texttt{#1}\index{#1 package@\texttt{#1} package}\index{packages!#1@\texttt{#1}}}% package name
\newcommand{\doccls}[1]{\texttt{#1}}% document class name
\newcommand{\docclsopt}[1]{\texttt{#1}\index{#1 class option@\texttt{#1} class option}\index{class options!#1@\texttt{#1}}}% document class option name
\newcommand{\docclsoptdef}[1]{\hlred{\texttt{#1}}\label{clsopt:#1}\index{#1 class option@\texttt{#1} class option}\index{class options!#1@\texttt{#1}}}% document class option name defined
\newcommand{\docmsg}[2]{\bigskip\begin{fullwidth}\noindent\ttfamily#1\end{fullwidth}\medskip\par\noindent#2}
\newcommand{\docfilehook}[2]{\texttt{#1}\index{file hooks!#2}\index{#1@\texttt{#1}}}
\newcommand{\doccounter}[1]{\texttt{#1}\index{#1 counter@\texttt{#1} counter}}


\usepackage{url}
\usepackage{listings}
\usepackage{color}
\usepackage{courier}
\usepackage{graphics}
\usepackage{caption}
\usepackage{textcomp}
\usepackage[printonlyused]{acronym}
%\usepackage{upquote}
%\usepackage{microtype}

\usepackage{textcomp} % additional fonts, required for upquote in listings
\usepackage{afterpage}
\usepackage{placeins} % FloatBarrier
\usepackage[detect-weight]{siunitx} % nice! SI units and print numbers

\usepackage{ifthen}
\usepackage{tikz}
\usepackage{neuralnetwork}

\usepackage{bm} % bold math

%\usepackage{lastpage}
\usepackage{palatino} 

\usepackage{tcolorbox} % for box-style TBC

\setcounter{section}{0}
\setcounter{secnumdepth}{6}


% commands not part of the style
\newcommand{\mypersec}{\si{\meter\per\second}}
\newcommand{\mympersecsq}{\si{\meter\per\second\squared}}
\newcommand{\mydegpersec}{\si{\degree\per\second}}
\newcommand{\mykm}{\si{\kilo\metre}}
\newcommand{\mym}{\si{\metre}}
\newcommand{\mymm}{\si{\milli\metre}}
\newcommand{\myum}{\si{\micro\metre}}
\newcommand{\myms}{\si{\milli\second}}
\newcommand{\myus}{\si{\micro\second}}
\newcommand{\myuw}{\si{\micro\watt}}
\newcommand{\myua}{\si{\micro\ampere}}
\newcommand{\myusr}{\si{\micro\steradian}}
\newcommand{\mydegree}{\si{\degree}}
\newcommand{\mymthreedb}{$-$3~dB}
\newcommand{\Unl}[1]{\ensuremath{\underline{#1}}}

\definecolor{LightGrey}{rgb}{0.96,0.96,0.96}
\definecolor{LightGrey}{rgb}{0.95,0.95,0.95}
\definecolor{light-gray}{gray}{0.95}
\definecolor{half-gray}{gray}{0.75}
\definecolor{LightRed}{rgb}{1.0,0.9,0.9}

\newcommand{\colheightrule}{\rule[-2mm]{0mm}{6.5mm}}
\newcommand{\marginnotecjw}[1]{ \marginpar{{\scriptsize #1}}}
\newcommand{\CJWpar}[2]{\parbox{#1}{\rule{0cm}{4mm}#2\rule[-2mm]{0cm}{4mm}}}
\newcommand{\HiLite}[1]{\fcolorbox{red}{red}{#1}}
\newcommand{\res}{\marginpar{$\ast$}}
\newcommand{\spec}[1]{\fcolorbox{half-gray}{light-gray}{#1}}

\newcommand{\TBC}[1]{%
    \par
    \noindent
    \begin{minipage}{1.0\linewidth}
      \begin{tcolorbox}[colback=purple!5, colframe=purple!95!black, coltext=black,
        coltitle=white, fonttitle=\bfseries, title={To be completed},%
        left=0mm, right=0mm, top=0mm, bottom=0mm,%
        before=\vspace{2pt}, after=\vspace{2pt}]
        % \setlength{\parskip}{0.5em} \vspace{-0.5em}
        %\setlength{\parindent}{1em} \noindent
        #1
        %\setlength{\parindent}{0em} \noindent
      \end{tcolorbox}
    \end{minipage}
}

\newcommand*\diff{\mathop{}\!\mathrm{d}}
\newcommand*\Diff[1]{\mathop{}\!\mathrm{d^#1}}
\newcommand*\euler{\mathop{}\!\mathrm{e}}
\newcommand*\imag{\mathop{}\!\mathrm{j}}

\newcommand{\mth}[1]{$#1$\textsuperscript{th}\ }

% set up the listings environment
%\begin{lstlisting}[hkey=value list]
%code here
%\end{lstlisting}
%\lstinline[hkey=value list]<character>source code<same character>
%\lstinputlisting[lastline=4]{listings.sty}

%\lstdefinelanguage{none}{identifierstyle=}

% set up listings package details
\lstloadlanguages{TeX,C++,XML,Python}


%\lstset{ %
\lstdefinestyle{ossimstyle}{
language=,                % choose the language of the code
upquote=true, % gives the upquote instead of the curly quote
basicstyle=\ttfamily\footnotesize,       % the size of the fonts that are used for the code
                  % alternative ((basicstyle=\ttfamily\fontsize{7}{10}\selectfont]))
numbers=none,                   % where to put the line-numbers (normally left)
numberstyle=\tiny,
%stepnumber=2,
numbersep=5pt,
showspaces=false,               % show spaces adding particular underscores
showstringspaces=false,         % underline spaces within strings
showtabs=false,                 % show tabs within strings adding particular underscores
breaklines=true,        % sets automatic line breaking
breakatwhitespace=false,    % sets if automatic breaks should only happen at whitespace
extendedchars=true,
keywordstyle=\color{red},
prebreak=\raisebox{0ex}[0ex][0ex]{$\dlsh$}, % add linebreak symbol
captionpos=b,                   % sets the caption-position to bottom
frame=none,                   % 'lines' or 'none' adds a frame around the code
%backgroundcolor=\color{LightGrey},
tabsize=2,              % sets default tabsize to 2 spaces
%escapeinside={\%}{)},          % if you want to add a comment within your code
%escapeinside={\%*}{*)},          % if you want to add LaTeX within your code
%backgroundcolor=\color{LightGrey},  % choose the background color,  add \usepackage{color}
framesep=1pt,
xleftmargin=0pt,
xrightmargin=0pt,
captionpos=t,                    % sets the caption-position to top
%deletekeywords={...},            % if you want to delete keywords from the given language
numberbychapter=false,
}

\DeclareCaptionFont{white}{\color{white}}
\DeclareCaptionFormat{listing}{\colorbox[cmyk]{0.43, 0.35, 0.35,0.01}{\parbox{\textwidth}{\hspace{15pt}#1#2#3}}}
\captionsetup[lstlisting]{format=listing,labelfont=white,textfont=white, singlelinecheck=false, margin=0pt, font={bf,footnotesize}}

% make inline listing larger than display
%https://latex.org/forum/viewtopic.php?t=2072
\makeatletter
\lst@AddToHook{TextStyle}{\let\lst@basicstyle\ttfamily\footnotesize\fontfamily{pcr}\selectfont}
\makeatother

\lstset{style=ossimstyle}


\lstdefinestyle{pythonstyle}{
  backgroundcolor=\color{LightGrey},  % choose the background colour,  add \usepackage{color}
  language=Python,
}

\lstdefinestyle{cppstyle}{
  backgroundcolor=\color{LightGrey},  % choose the background colour,  add \usepackage{color}
  language=C++,
}

\lstdefinestyle{xmlstyle}{
  backgroundcolor=\color{LightGrey},  % choose the background colour,  add \usepackage{color}
  language=XML,
}

\lstdefinestyle{nonestyle}{
  backgroundcolor=\color{LightGrey},  % choose the background colour,  add \usepackage{color}
}

\renewcommand{\arraystretch}{1.5}

\newcommand{\myhline}{\ \hrulefill \ }

%\minipsize{90mm}{some text}
\newcommand{\minipsize}[2]{\begin{minipage}[t]{#1}\begin{flushleft}#2\end{flushleft}\end{minipage}}


%%%%%%%%%%%%%%%%%%%%%%%%%%%%%%%%%%%%%%%%%%%%%%%%%%%%%%%%%%%%%
%\usepackage{etoolbox}
%\makeatletter
%\patchcmd{\thebibliography}{%
%  \chapter{\bibname}\@mkboth{\MakeUppercase\bibname}{\MakeUppercase\bibname}}{%
%  \chapter{References}}{}{}
%\makeatother

%\bibliographystyle{plain}
%\bibliographystyle{acm}
\bibliographystyle{bib/dpsscjw}

\newcommand{\tabrule}{\rule[-2mm]{0mm}{6mm}}

%% Symbols used in missile dynamics
%% See ../Missile Airframe-Guidance Models/c03_01.tex for definitions
\newcommand{\fax}{\ensuremath{f_\mathrm{ax}}}
\newcommand{\fay}{\ensuremath{f_\mathrm{ay}}}
\newcommand{\faz}{\ensuremath{f_\mathrm{az}}}
\newcommand{\fth}{\ensuremath{f_\mathrm{th}}}
\newcommand{\ribi}{\ensuremath{r_{ib}^i}}
\newcommand{\vibi}{\ensuremath{v_{ib}^i}}
\newcommand{\cib}{\ensuremath{C_i^b}}
\newcommand{\wibb}{\ensuremath{\omega_{ib}^b}}
\newcommand{\Wibb}{\ensuremath{\Omega_{ib}^b}}
\newcommand{\qdyn}{\ensuremath{q_\mathrm{dyn}}}
\newcommand{\sref}{\ensuremath{S_\mathrm{ref}}}
\newcommand{\vsound}{\ensuremath{v_\mathrm{s}}}
\newcommand{\nbt}{\ensuremath{\hat{t}}}
\newcommand{\tburn}{\ensuremath{t_\mathrm{burn}}}
\newcommand{\tign}{\ensuremath{t_\mathrm{ign}}}
\newcommand{\alphamax}{\ensuremath{\alpha_\mathrm{max}}}
\newcommand{\cazero}{\ensuremath{C_{A_0}}}
\newcommand{\cabase}{\ensuremath{C_{A_\mathrm{base}}}}
\newcommand{\cainc}{\ensuremath{C_{A_i}}}
\newcommand{\caincfull}{\ensuremath{C_{A_{if}}}}
\newcommand{\caincempty}{\ensuremath{C_{A_{ie}}}}
\newcommand{\cn}{\ensuremath{C_N}}
\newcommand{\cnfull}{\ensuremath{C_{N_f}}}
\newcommand{\cnempty}{\ensuremath{C_{N_e}}}
\newcommand{\xref}{\ensuremath{X_\mathrm{ref}}}
\newcommand{\czalpha}{\ensuremath{{C_{Z_\alpha}}}}
\newcommand{\chk}{$\times$}



\newcommand{\formatpart}[1]{%
\begin{fullwidth}
    \centering%
    \partname~\thepart%
    \newline%
    \ \newline%
    \includegraphics[width=.7\textwidth,]{pic/neuronstylised}%
    \ \newline%
    \ \newline%
    #1%
\end{fullwidth}
    }

\titleformat% Formatting the part header
  {\part} % command
  [block] % shape
  {\bfseries\sc\Huge} % format
  {} % label
  {0pt} % sep
  { \formatpart
}%

%the following is required for carriage return symbol
%ftp://ftp.botik.ru/rented/znamensk/CTAN/fonts/mathabx/texinputs/mathabx.dcl
%https://secure.kitserve.org.uk/content/mathabx-font-symbol-redefinition-clash-latex
\DeclareFontFamily{U}{mathb}{\hyphenchar\font45}
\DeclareFontShape{U}{mathb}{m}{n}{
      <5> <6> <7> <8> <9> <10> gen * mathb
      <10.95> mathb10 <12> <14.4> <17.28> <20.74> <24.88> mathb12
      }{}
\DeclareSymbolFont{mathb}{U}{mathb}{m}{n}
\DeclareMathSymbol{\dlsh}{3}{mathb}{"EA}


%%%%%%%%%%%%%%%%%%%%%%%%%%%%%%%%%%%%%%%%%%%%%%%%%%%%%%%%%%%%%%%%%%%
% files making up this document

\includeonly{cf00,c00,c01,c02,c03,c04,c05,c06,c07
,c08,c09,c10,c11,c12,c13
,c0a,c0b,c0c,c0d,cbib
,p04c01,p04c02,p04c03}

%\includeonly{p04c02}
%\includeonly{cf00}

% Generates the index
\usepackage{makeidx}
\makeindex

\begin{document}


% -*- TeX -*- -*- UK -*- -*- Soft -*-

% Front matter
\frontmatter

% r.1 blank page
\blankpage


% r.3 full title page
\maketitle


% v.4 copyright page
\newpage
\begin{fullwidth}
~\vfill
\thispagestyle{empty}
\setlength{\parindent}{0pt}
\setlength{\parskip}{\baselineskip}
Copyright \copyright\ \the\year\ \thanklessauthor, as indicated in the text

%\par\smallcaps{Published by \thanklesspublisher}

\par\smallcaps{https://github.com/NelisW/NeuralNetworks-DeepLearning-Notes}

\par
This work is licensed under a Creative Commons Attribution-
NonCommercial 4.0 Unported License. This means you’re free to
copy, share, and build on this book, but not to sell it. 
You may obtain a copy
of the License at \url{https://creativecommons.org/licenses/by-nc/4.0/}. Unless
required by applicable law or agreed to in writing, software distributed
under the License is distributed on an \smallcaps{``AS IS'' BASIS, WITHOUT
WARRANTIES OR CONDITIONS OF ANY KIND}, either express or implied. See the
License for the specific language governing permissions and limitations
under the License.\index{license}
If you’re interested in commercial use, please contact both the authors.

\par\textit{Current printing, \monthyear}
\end{fullwidth}

% r.5 contents
\tableofcontents

\listoffigures

%\listoftables

\input{abbrev.tex}


%%
% Start the main matter (normal chapters)
\mainmatter

\chapter*{Introduction}

The purpose with this book is to document some of my learning about \ac{ML}.
This document does not attempt to add any new knowledge to the field, there are several other resources with strong mathematics available for this purpose \cite{geron2017handson,Webb2002statpatn,Michie94,theodoridis2003,Duda2001,Bishop1995,Bishop2006,Goodfellow2016}. There is also a huge body of knowledge on the internet, most of it of introductory or review nature, but very useful to gain a general (non-mathematical) understanding.
The focus here is on exploring the concepts and explaining these concepts to gain an intuitive but practical understanding.

The book has three distinctly separate parts: it contains Michael Nielsen's book as Part I, my own notes in Part II, and some more information from diverse sources in Part III.  The material in Part I is presented with minimal change so as to retain Michael's excellent review content.  The material in Part II is part original and part a rehash of other people's work. The material in Part III is mostly a rehash of other people's work.

\section*{Terminology}

\begin{marginfigure}
\includegraphics{chapterf00-01}
\end{marginfigure}

The newcomer can easily be confused by the terminology used in the field.  There is some differences in the details of different definitions, but the broad consensus can be depicted as fields of expertise within fields of expertise, often drawn as a Venn diagram as shown here.

\ac{AI} can be broadly defined as the simulation of human intelligence by machines, including computers. These \ac{AI} processes include learning, reasoning and self-correction. Work in this area started as early as the 1940s. \ac{ML}\cite{WikiPediaMachineLearning2019,DanielFaggella2019}, is normally defined as the application of \ac{AI} to learn from data without explicitly programming the learned behaviour. \ac{ML} therefore focuses on the development of programs that can access data and learn new behaviour from the data, by themselves.  Early ML work started in the 1980s. Arguably the neural networks subset does not belong to the figure, but this technology plays such a key role that perhaps it should earn to be explicitly shown.  Neural nets is a computing technique loosely modelled on the human brain, that can  recognise patterns.  These neural nets are a dominant part of the bigger machine learning field. Neural nets attracted much attention in the 1980's and 1990s, although it was used extensively, the application of this technology was limited by computing power.  In particular, the pattern information had to be carefully prepared for the neural net, which sometimes required very special skills. Practical application of neural nets were limited to three-layer neural nets, which could be trained by the then-available algorithms and computers.  \ac{DLea}\cite{WikiPediaDeepLearning2019}, which required more than three layer neural nets, with deeply hidden layers (hence the term deep learning) was defined and understood already in the 1980s.  The algorithms and computers could not solve the deep learning problems at the time.   Around 2006 new algorithms were was proposed that opened up the deep learning neural nets to practical use.
Deep learning is therefore a special application of neural nets, which is a part of machine learning, which in turn is part of artificial intelligence.

\newthought{Machine learning} is defined as\cite{DanielFaggella2019} ``Machine Learning is the science of getting computers to learn and act like humans do, and improve their learning over time in autonomous fashion, by feeding them data and information in the form of observations and real-world interactions.''


\section*{Historic Perspective}

In his excellent blog post \cite{Vazquez2018} Favio V\'{a}zquez gives a brief introduction to the history of \ac{DL} with a very nice historic time line (Figure~\ref{fig:deeplearningtimeline}).

\begin{figure*}[tph]
\includegraphics[width=\textwidth]{deeplearningtimeline}
\caption{Favio V\'{a}zquez's \ac{DL} time line}
\label{fig:deeplearningtimeline}
\end{figure*}



%\FloatBarrier

\section*{The Plan}

My present motivated plan to learn more about \ac{ML} is as shown below. This is the third plan since I started, which demonstrates the volatility of the \ac{ML} world. What seems like a good idea today is in insanely poor choice six months later.

\begin{enumerate}
\item \marginnote{A study \cite{Ericsson93therole} found that full mastery requires long and consistent practice, but another study \cite{Macnamara2014} found that practice alone is not a guarantee for success.} Select the technology/product with utmost care.  Provided you have the ability, it takes ten years or ten thousand hours to fully master a topic. How many ten-year cycles can you afford in your lifetime?

\item \marginnote{Even staying with large companies is risky. Google is notorious for dropping services. Do you remember the MFC technology from Microsoft? It is interesting that Qt, HTML, Python and JavaScript are still around after all these years.} You want to invest in a technology with a long shelf life (probably not possible in \ac{ML}). Stay with mainstream and large projects with a wide following. Stay away from small or risky projects, time spent there is time wasted. 
    
\item I want to stay on the Windows platform this is where all my tools are (I do not have a Linux boot PC at present).  This requirement severely limits my options (see below). Linux is miles ahead here, most of the serious work is done on Linux.
    
\item Stay local, not in the cloud.   If you have large models with large data sets, using the cloud services is far better, but I have no funding to pay for cloud services at this time.

\item Stay with open source projects with large following and a good Internet support base. The bigger the user base, the better you protect your investment in terms of support and long term survivability. Services such as StackOverflow provide more useful support than does commercial companies, simply because of the large support base. People often denigrate the value of YouTube, but there are also some seriously useful tutorials and support information there. 

\item Steps:
\begin{enumerate}
\item Start with Michael Nielsen's excellent introductory online book \cite{Nielsen2015}. His book is included as Part~I of this document.  The purpose with the book is to provide understanding at a conceptual level, not as a rigourous academic treatise; which is what I need to start of with.
\item Start with Scikit Learn, simply because it is easily installable, does not require a \ac{GPU} software installation, has a large example base and is reasonably well documented.
\item Invest in learning TensorFlow \cite{TensorFlow2Alpha2019}.  I previously considered PyTorch \cite{paszkePyTorch2017,PyTorch2019} and fastai \cite{fastai2019}, but found it difficult to do a GPU install on Windows.  One of the objections against TensorFlow~1 was its awkward \lstinline{sessions}  construction, but this falls away with TensorFlow~2.  Google also integrated Keras \cite{cholletkeras2015,cholletkerasio2015} into TensorFlow~2 as a native higher-level framework.
    Installation of GPU TensorFlow~2 is still not simple for Windows, but I will start with the \ac{CPU} version.
\item Work through G\'{e}ron's book \cite{geron2017handson} \textit{Hands-on machine learning with Scikit-Learn and TensorFlow}.  The first part of the book uses Scikit Learn to establish a good understanding and then moves on to using the TensorFlow tools.
    The book receives good reviews and is presently under revision for a second edition, covering TensorFlow~2.
\item I want to document my learning process in Part~II of this book, and add insights from other authors into Part~III of this book. This will come as time availability allows.
\item For in-depth backup and if I ever need this, I will consult the academic books \cite{geron2017handson,Webb2002statpatn,Michie94,theodoridis2003,Duda2001,Bishop1995,Bishop2006,Goodfellow2016}, but for now, my interest is still at an elementary level.
\end{enumerate}
\item All of the above will be done at a leisurely pace outside of working hours and jointly with my life partner and fellow traveller Riana.
\end{enumerate}


\section*{Attribution}

\begin{enumerate}
\item 
The entirety of Part I is taken from Michael Nielsen's online book\cite{Nielsen2015}. Some minor changes were made to the text.

\item
The stylistic neuron on the Part header pages were adapted from \cite{Erler2004}. 

\item
Some of the neural diagrams  are created with the \lstinline{neural} TikZ module for \LaTeX{}\cite{cowan2019} by Mark Cowan.
\end{enumerate}
 % front matter

%\part{Neural Networks and Deep Learning: Michael Nielsen}
\include{c00} % Nielsen front matter
\include{c01} %[[Done]] Neural Nets Recognizing Handwritten Digits 
% -*- TeX -*- -*- UK -*- -*- Soft -*-

\chapter{How the Backpropagation Algorithm Works}


The material in this chapter is taken from\\
\lstinline{http://neuralnetworksanddeeplearning.com/chap2.html}

\newthought{Backpropagation.}
In the last chapter we saw how neural networks can learn their weights and biases using the gradient descent algorithm. There was, however, a gap in our explanation: we didn't discuss how to compute the gradient of the cost function. That's quite a gap! In this chapter I'll explain a fast algorithm for computing such gradients, an algorithm known as \textit{backpropagation}. 


The backpropagation algorithm was originally introduced in the 1970s, but its importance wasn't fully appreciated until a famous 1986 paper \cite{rumelhart1986} by David Rumelhart, Geoffrey Hinton\footnote{Hinton's home page \cite{Hinton2019}.}, and Ronald Williams. That paper describes several neural networks where backpropagation works far faster than earlier approaches to learning, making it possible to use neural nets to solve problems which had previously been insoluble. Today, the backpropagation algorithm is the workhorse of learning in neural networks.


This chapter is more mathematically involved than the rest of the book. If you're not crazy about mathematics you may be tempted to skip the chapter, and to treat backpropagation as a black box whose details you're willing to ignore. Why take the time to study those details?

The reason, of course, is understanding. At the heart of backpropagation is an expression for the partial derivative $\partial C / \partial w$ of the cost function $C$ with respect to any weight $w$ (or bias $b$) in the network. The expression tells us how quickly the cost changes when we change the weights and biases. And while the expression is somewhat complex, it also has a beauty to it, with each element having a natural, intuitive interpretation. And so backpropagation isn't just a fast algorithm for learning. It actually gives us detailed insights into how changing the weights and biases changes the overall behaviour of the network. That's well worth studying in detail.

With that said, if you want to skim the chapter, or jump straight to the next chapter, that's fine. I've written the rest of the book to be accessible even if you treat backpropagation as a black box. There are, of course, points later in the book where I refer back to results from this chapter. But at those points you should still be able to understand the main conclusions, even if you don't follow all the reasoning.

\section{A fast matrix-based approach to computing the output from a neural network}

Before discussing backpropagation, let's warm up with a fast matrix-based algorithm to compute the output from a neural network. We actually already briefly saw this algorithm near the end of the last chapter, but I described it quickly, so it's worth revisiting in detail. In particular, this is a good way of getting comfortable with the notation used in backpropagation, in a familiar context.

Let's begin with a notation which lets us refer to weights in the network in an unambiguous way. We'll use $w^l_{jk}$ to denote the weight for the connection from the \mth{k} neuron in the \mth{l-1} layer to the \mth{j} neuron in the \mth{l} layer. So, for example, the diagram on the right shows the weight on a connection from the fourth neuron in the second layer to the second neuron in the third layer of a network.

\begin{marginfigure}
\includegraphics{chapter02-01}
\end{marginfigure}

This notation is cumbersome at first, and it does take some work to master. But with a little effort you'll find the notation becomes easy and natural. One quirk of the notation is the ordering of the $j$ and $k$ indices. You might think that it makes more sense to use $j$ to refer to the input neuron, and k to the output neuron, not vice versa, as is actually done. I'll explain the reason for this quirk below.


We use a similar notation for the network's biases and activations. Explicitly, we use $b^l_j$ for the bias of the \mth{j} neuron in the \mth{l} layer. And we use $a^l_j$ for the activation of the \mth{j} neuron in the \mth{l} layer. The next diagram shows examples of these notations in use.

\begin{marginfigure}
\includegraphics{chapter02-02}
\end{marginfigure}


With these notations, the activation $a^l_j$ of the \mth{j} neuron in the \mth{l} layer is related to the activations in the \mth{l-1} layer by the equation (compare Equation~\ref{eq:ch01-04-sigmoidoutput} and surrounding discussion in the last chapter) 

\begin{equation}
a_{j}^{l}=\sigma\left(\sum_{k} w_{j k}^{l} a_{k}^{l-1}+b_{j}^{l}\right)
\label{eq:c02-23}
\end{equation}

where the sum is over all neurons $k$ in the \mth{l-1} layer. To rewrite this expression in a matrix form we define a \textit{weight matrix} $w^l$ for each layer, $l$. The entries of the weight matrix $w^l$ are just the weights connecting to the \mth{l} layer of neurons, that is, the entry in the \mth{j} row and \mth{k} column is $w^l_{jk}$. Similarly, for each layer $l$ we define a \textit{bias vector}, $b^l$. You can probably guess how this works - the components of the bias vector are just the values $b^l_j$, one component for each neuron in the \mth{l} layer. And finally, we define an \textit{activation vector} al whose components are the activations $a^l_j$.

The last ingredient we need to rewrite Equation~\ref{eq:c02-23} in a matrix form is the idea of vectorizing a function such as $\sigma$. We met vectorization briefly in the last chapter, but to recap, the idea is that we want to apply a function such as $\sigma$ to every element in a vector $v$. We use the obvious notation $\sigma(v)$ to denote this kind of elementwise application of a function. That is, the components of $\sigma(v)$ are just $\sigma(v)_j=\sigma(v_j)$. As an example, if we have the function $f(x)=x^2$ then the vectorized form of $f$ has the effect 
\begin{equation}
f\left(\left[ \begin{array}{l}{2} \\ {3}\end{array}\right]\right)
=
\left[ \begin{array}{l}{f(2)} \\ {f(3)}\end{array}\right]
=
\left[ \begin{array}{l}{4} \\ {9}\end{array}\right]
\label{eq:c02-24}
\end{equation}
that is, the vectorized $f$ just squares every element of the vector.

With these notations in mind, Equation~\ref{eq:c02-23} can be rewritten in the beautiful and compact vectorized form 
\begin{equation}
a^{l}=\sigma\left(w^{l} a^{l-1}+b^{l}\right)
\label{eq:c02-25}
\end{equation}
This expression gives us a much more global way of thinking about how the activations in one layer relate to activations in the previous layer: we just apply the weight matrix to the activations, then add the bias vector, and finally apply the $\sigma$ function\footnote{By the way, it's this expression that motivates the quirk in the $w^l_{jk}$ notation mentioned earlier. If we used $j$ to index the input neuron, and $k$ to index the output neuron, then we'd need to replace the weight matrix in Equation~\ref{eq:c02-25} by the transpose of the weight matrix. That's a small change, but annoying, and we'd lose the easy simplicity of saying (and thinking) ``apply the weight matrix to the activations''.}.

That global view is often easier and more succinct (and involves fewer indices!) than the neuron-by-neuron view we've taken to now. Think of it as a way of escaping index hell, while remaining precise about what's going on. The expression is also useful in practice, because most matrix libraries provide fast ways of implementing matrix multiplication, vector addition, and vectorization. Indeed, the code in Section~\ref{sec:Implementingournetworktoclassifydigits} in the last chapter made implicit use of this expression to compute the behaviour of the network.

When using Equation~\ref{eq:c02-25} to compute $a^l$, we compute the intermediate quantity $z^{l} \equiv w^{l} a^{l-1}+b^{l}$ along the way. This quantity turns out to be useful enough to be worth naming: we call $z^l$ the weighted input to the neurons in layer $l$. We'll make considerable use of the weighted input $z^l$ later in the chapter. Equation~\ref{eq:c02-25} is sometimes written in terms of the weighted input, as $a^{l}=\sigma\left(z^{l}\right)$. It's also worth noting that $z^l$ has components , that is, $z_l^j$ is just the weighted input to the activation function for neuron $j$ in layer $l$.

\section{The two assumptions we need about the cost function}

The goal of backpropagation is to compute the partial derivatives $\partial C / \partial w$ and $\partial C / \partial b$ of the cost function $C$ with respect to any weight $w$ or bias $b$ in the network. For backpropagation to work we need to make two main assumptions about the form of the cost function. Before stating those assumptions, though, it's useful to have an example cost function in mind. We'll use the quadratic cost function from last chapter (c.f. Equation~\ref{eq:c01-06-MSEcostfunction}). In the notation of the last section, the quadratic cost has the form 


\begin{equation}
C=\frac{1}{2 n} \sum_{x}\parallel y(x)-a^{L}(x)\parallel ^{2}
\label{eq:c02-26}
\end{equation}
where: $n$ is the total number of training examples; the sum is over individual training examples, $x$; $y=y(x)$ is the corresponding desired output; $L$ denotes the number of layers in the network; and $a^L=a^L(x)$ is the vector of activations output from the network when $x$ is input.

Okay, so what assumptions do we need to make about our cost function, $C$, in order that backpropagation can be applied? The first assumption we need is that the cost function can be written as an average $C=\frac{1}{n} \sum_{x} C_{x}$ over cost functions $C_x$ for individual training examples, $x$. This is the case for the quadratic cost function, where the cost for a single training example is  
$C_{x}=\frac{1}{2}\parallel y-a^{L}\parallel^{2}$.  This assumption will also hold true for all the other cost functions we'll meet in this book.

The reason we need this assumption is because what backpropagation actually lets us do is compute the partial derivatives$\partial C_{x} / \partial w$ and $\partial C_{x} / \partial b$ for a single training example. We then recover $\partial C / \partial w$ and $\partial C / \partial b$ by averaging over training examples. In fact, with this assumption in mind, we'll suppose the training example $x$ has been fixed, and drop the $x$ subscript, writing the cost $C_x$ as $C$. We'll eventually put the $x$ back in, but for now it's a notational nuisance that is better left implicit.

The second assumption we make about the cost is that it can be written as a function of the outputs from the neural network as shown in the figure.
\begin{marginfigure}
\includegraphics{chapter02-03}
\end{marginfigure}


For example, the quadratic cost function satisfies this requirement, since the quadratic cost for a single training example $x$ may be written as 
\begin{equation}
C=\frac{1}{2}\left\|y-a^{L}\right\|^{2}=\frac{1}{2} \sum_{j}\left(y_{j}-a_{j}^{L}\right)^{2}
\label{eq:c02-27}
\end{equation}
and thus is a function of the output activations. Of course, this cost function also depends on the desired output $y$, and you may wonder why we're not regarding the cost also as a function of $y$. Remember, though, that the input training example $x$ is fixed, and so the output $y$ is also a fixed parameter. In particular, it's not something we can modify by changing the weights and biases in any way, i.e., it's not something which the neural network learns. And so it makes sense to regard $C$ as a function of the output activations $a^L$ alone, with $y$ merely a parameter that helps define that function.


\section{The Hadamard product, $s\odot t$}

The backpropagation algorithm is based on common linear algebraic operations - things like vector addition, multiplying a vector by a matrix, and so on. But one of the operations is a little less commonly used. In particular, suppose $s$
and $t$ are two vectors of the same dimension. Then we use $s\odot t$ to denote the elementwise product of the two vectors. Thus the components of $s\odot t$ are just $(s \odot t)_{j}=s_{j} t_{j}$. As an example, 

\begin{equation}
\left[ \begin{array}{l}{1} \\ {2}\end{array}\right] \odot \left[ \begin{array}{l}{3} \\ {4}\end{array}\right]=\left[ \begin{array}{l}{1 * 3} \\ {2 * 4}\end{array}\right]=\left[ \begin{array}{l}{3} \\ {8}\end{array}\right]
\label{eq:c02-28}
\end{equation}

This kind of elementwise multiplication is sometimes called the \textit{Hadamard product} or \textit{Schur product}. We'll refer to it as the Hadamard product. Good matrix libraries usually provide fast implementations of the Hadamard product, and that comes in handy when implementing backpropagation.


\section{The four fundamental equations behind backpropagation}

Backpropagation is about understanding how changing the weights and biases in a network changes the cost function. Ultimately, this means computing the partial derivatives $\partial C / \partial w_{j k}^{l}$ and $\partial C / \partial b_{j}^{l}$. But to compute those, we first introduce an intermediate quantity, $\delta_{j}^{l}$, which we call the \textit{error} in the \mth{j} neuron in the \mth{l} layer. 

Backpropagation will give us a procedure to compute the error $\delta^{l}_j$, and then will relate $\delta_{j}^{l}$ to $\partial C / \partial w_{j k}^{l}$ and $\partial C / \partial b_{j}^{l}$.

\begin{marginfigure}
\includegraphics{chapter02-04}
\end{marginfigure}

To understand how the error is defined, imagine there is a demon in our neural network.
The demon sits at the \mth{j} neuron in layer $l$. As the input to the neuron comes in, the demon messes with the neuron's operation. It adds a little change $\Delta z_{j}^{l}$ to the neuron's weighted input, so that instead of outputting $\sigma\left(z_{j}^{l}\right)$, the neuron instead outputs $\sigma\left(z_{j}^{l}+\Delta z_{j}^{l}\right)$. This change propagates through later layers in the network, finally causing the overall cost to change by an amount $\frac{\partial C}{\partial z_{j}^{l}} \Delta z_{j}^{l}$.

Now, this demon is a good demon, and is trying to help you improve the cost, i.e., they're trying to find a $\Delta z_{j}^{l}$
which makes the cost smaller. Suppose $\frac{\partial C}{\partial z_{j}^{l}}$ has a large value (either positive or negative). Then the demon can lower the cost quite a bit by choosing $\Delta z_{j}^{l}$ to have the opposite sign to $\frac{\partial C}{\partial z_{j}^{l}}$. By contrast, if $\frac{\partial C}{\partial z_{j}^{l}}$ is close to zero, then the demon can't improve the cost much at all by perturbing the weighted input $ z_{j}^{l}$. So far as the demon can tell, the neuron is already pretty near optimal\footnote{This is only the case for small changes $\Delta z_{j}^{l}$, of course. We'll assume that the demon is constrained to make such small changes.}. And so there's a heuristic sense in which $\frac{\partial C}{\partial z_{j}^{l}}$
is a measure of the error in the neuron.

Motivated by this story, we define the error $\delta_{j}^{l}$ of neuron $j$ in layer $l$ by 
\begin{equation}
\delta_{j}^{l} \equiv \frac{\partial C}{\partial z_{j}^{l}}
\label{eq:c02-29}
\end{equation}

As per our usual conventions, we use $\delta^{l}$ to denote the vector of errors associated with layer $l$. Backpropagation will give us a way of computing  $\delta^{l}$  for every layer, and then relating those errors to the quantities of real interest, $\partial C / \partial w_{j k}^{l}$ and $\partial C / \partial b_{j}^{l}$.

You might wonder why the demon is changing the weighted input $z_{j}^{l}$. Surely it'd be more natural to imagine the demon changing the output activation $a^l_j$, with the result that we'd be using $\frac{\partial C}{\partial a_{j}^{l}}$ as our measure of error. In fact, if you do this things work out quite similarly to the discussion below. But it turns out to make the presentation of backpropagation a little more algebraically complicated. So we'll stick with $\delta_{j}^{l}=\frac{\partial C}{\partial z_{j}^{l}}$ as our measure of error\footnote{In classification problems like MNIST the term ``error'' is sometimes used to mean the classification failure rate. E.g., if the neural net correctly classifies 96.0 percent of the digits, then the error is 4.0 percent. Obviously, this has quite a different meaning from our $\delta$ vectors. In practice, you shouldn't have trouble telling which meaning is intended in any given usage.}.

\textbf{Plan of attack:} Backpropagation is based around four fundamental equations. Together, those equations give us a way of computing both the error $\delta^{l}$ and the gradient of the cost function. I state the four equations below. Be warned, though: you shouldn't expect to instantaneously assimilate the equations. Such an expectation will lead to disappointment. In fact, the backpropagation equations are so rich that understanding them well requires considerable time and patience as you gradually delve deeper into the equations. The good news is that such patience is repaid many times over. And so the discussion in this section is merely a beginning, helping you on the way to a thorough understanding of the equations.

Here's a preview of the ways we'll delve more deeply into the equations later in the chapter: I'll give a short proof of the equations (Section~\ref{sec:Proofofthefourfundamentalequations(optional)}), which helps explain why they are true; we'll restate the equations (Section~\ref{sec:Thebackpropagationalgorithm}) in algorithmic form as pseudocode (Section~\ref{sec:Thecodeforbackpropagation}), and see how the pseudocode can be implemented as real, running Python code; and, in the final section of the chapter (Section~\ref{sec:Backpropagation:thebigpicture}), we'll develop an intuitive picture of what the backpropagation equations mean, and how someone might discover them from scratch. Along the way we'll return repeatedly to the four fundamental equations, and as you deepen your understanding those equations will come to seem comfortable and, perhaps, even beautiful and natural.


\textbf{An equation for the error in the output layer, $\delta^{l}$:} The components of $\delta^{l}$ are given by 
\begin{equation}
\delta_{j}^{L}=\frac{\partial C}{\partial a_{j}^{L}} \sigma^{\prime}\left(z_{j}^{L}\right)
\label{eq:c02-BP1}
\end{equation}

This is a very natural expression. The first term on the right, $\partial C / \partial a_{j}^{L}$, just measures how fast the cost is changing as a function of the \mth{j} output activation. If, for example, $C$ doesn't depend much on a particular output neuron, $j$, then $\Delta_{j}^{L}$ will be small, which is what we'd expect. The second term on the right, $\sigma^{\prime}\left(z_{j}^{L}\right)$, measures how fast the activation function $\sigma$  is changing at $z_{j}^{L}$.

Notice that everything in Equation~\ref{eq:c02-BP1} is easily computed. In particular, we compute $z_{j}^{L}$ while computing the behaviour of the network, and it's only a small additional overhead to compute $\sigma^{\prime}\left(z_{j}^{L}\right)$. The exact form of $\partial C / \partial a_{j}^{L}$ will, of course, depend on the form of the cost function. However, provided the cost function is known there should be little trouble computing  $\partial C / \partial a_{j}^{L}$. For example, if we're using the quadratic cost function then $C=\frac{1}{2} \sum_{j}\left(y_{j}-a_{j}^{L}\right)^{2}$, and so $\partial C / \partial a_{j}^{L}=\left(a_{j}^{L}-y_{j}\right)$, which obviously is easily computable.

Equation~\ref{eq:c02-BP1} is a componentwise expression for  $\delta^{l}$. It's a perfectly good expression, but not the matrix-based form we want for backpropagation. However, it's easy to rewrite the equation in a matrix-based form, as 
\begin{equation}
\delta^{l}=\nabla_{a} C \odot \sigma^{\prime}\left(z^{L}\right)
\label{eq:c02-BP1a}
\end{equation}
Here, $\nabla_a C$ is defined to be a vector whose components are the partial derivatives $\partial C / \partial a_{j}^{L}$. You can think of $\nabla_a C$  as expressing the rate of change of $C$ with respect to the output activations. It's easy to see that Equations~\ref{eq:c02-BP1} and \ref{eq:c02-BP1a} are equivalent, and for that reason from now on we'll use Equation~\ref{eq:c02-BP1} interchangeably to refer to both equations. As an example, in the case of the quadratic cost we have $\nabla_{a} C=\left(a^{L}-y\right)$, and so the fully matrix-based form of Equation~\ref{eq:c02-BP1} becomes 

\begin{equation}
\delta^{l}=\left(a^{L}-y\right) \odot \sigma^{\prime}\left(z^{L}\right)
\label{eq:c02-30}
\end{equation}

As you can see, everything in this expression has a nice vector form, and is easily computed using a library such as Numpy.

\textbf{An equation for the error $\delta^{l}$ in terms of the error in the next layer, $\delta^{l+1}$: }
In particular 
\begin{equation}
\delta^{l}=\left(\left(w^{l+1}\right)^{T} \delta^{l+1}\right) \odot \sigma^{\prime}\left(z^{l}\right)
\label{eq:c02-BP2}
\end{equation}
where $(w^{l+1})^T$ is the transpose of the weight matrix $w^{l+1}$ for the \mth{l+1} layer. This equation appears complicated, but each element has a nice interpretation. Suppose we know the error $\delta^{l+1}$ at the \mth{l+1} layer. When we apply the transpose weight matrix, $(w^{l+1})^T$, we can think intuitively of this as moving the error backward through the network, giving us some sort of measure of the error at the output of the \mth{l} layer. We then take the Hadamard product $\odot \sigma^{\prime}\left(z^{l}\right)$. This moves the error backward through the activation function in layer $l$, giving us the error $\delta^{l}$ in the weighted input to layer $l$.

By combining Equation~\ref{eq:c02-BP2} with Equation~\ref{eq:c02-BP1} we can compute the error $\delta^{l}$
for any layer in the network. We start by using Equation~\ref{eq:c02-BP1} to compute  $\delta^{L}$, then apply Equation~\ref{eq:c02-BP2} to compute  $\delta^{L-1}$, then Equation (BP2) again to compute  $\delta^{L-1}$, and so on, all the way back through the network.


\textbf{An equation for the rate of change of the cost with respect to any bias in the network:} In particular: 
\begin{equation}
\frac{\partial C}{\partial b_{j}^{l}}=\delta_{j}^{l}
\label{eq:c02-BP3}
\end{equation}
That is, the error $\delta_{j}^{L}$ is exactly equal to the rate of change $\partial C / \partial b_{j}^{l}$. This is great news, since Equation~\ref{eq:c02-BP1} and Equation~\ref{eq:c02-BP2} have already told us how to compute~$\delta^{l}_j$. We can rewrite Equation~\ref{eq:c02-BP3} in shorthand as 
\begin{equation}
\frac{\partial C}{\partial b}=\delta
\label{eq:c02-31}
\end{equation}
where it is understood that $\delta$ is being evaluated at the same neuron as the bias $b$.

\textbf{An equation for the rate of change of the cost with respect to any weight in the network:} 
In particular: 
\begin{equation}
\frac{\partial C}{\partial w_{j k}^{l}}=a_{k}^{l-1} \delta_{j}^{l}
\label{eq:c02-BP4}
\end{equation}
This tells us how to compute the partial derivatives $\partial C / \partial w_{j k}^{l}$ in terms of the quantities $\delta^{l}$ and $a^{l-1}$, which we already know how to compute. The equation can be rewritten in a less index-heavy notation as 
\begin{equation}
\frac{\partial C}{\partial w}=a_{\mathrm{in}} \delta_{\mathrm{out}}
\label{eq:c02-32}
\end{equation}
where it's understood that $a_\textrm{in}$ is the activation of the neuron input to the weight $w$, and $\delta_\textrm{out}$ is the error of the neuron output from the weight $w$. Zooming in to look at just the weight $w$, and the two neurons connected by that weight, we can depict this as shown in the picture.

\begin{marginfigure}
\includegraphics{chapter02-05}
\end{marginfigure}

A nice consequence of Equation~\ref{eq:c02-32} is that when the activation $a_\textrm{in}$ is small, $a_{\mathrm{in}} \approx 0$, the gradient term $\partial C / \partial w$ will also tend to be small. In this case, we'll say the weight \textit{learns slowly}, meaning that it's not changing much during gradient descent. In other words, one consequence of Equation~\ref{eq:c02-BP4} is that weights output from low-activation neurons learn slowly.

There are other insights along these lines which can be obtained from Equations~\ref{eq:c02-BP1} to \ref{eq:c02-BP4}. Let's start by looking at the output layer.  Consider the term $\sigma^{\prime}\left(z_{j}^{L}\right)$ in Equation~\ref{eq:c02-BP1}. Recall from the graph of the sigmoid function in the last chapter that the $\sigma$ function becomes very flat when $\sigma\left(z_{j}^{L}\right)$ is approximately 0 or 1. When this occurs we will have $\sigma^{\prime}\left(z_{j}^{L}\right) \approx 0$. And so the lesson is that a weight in the final layer will learn slowly if the output neuron is either low activation ($\approx 0 $) or high activation ($\approx 1$). In this case it's common to say the output neuron has \textit{saturated} and, as a result, the weight has stopped learning (or is learning slowly). Similar remarks hold also for the biases of output neuron.

We can obtain similar insights for earlier layers. In particular, note the $\sigma^{\prime}\left(z^{l}\right)$
term in Equation~\ref{eq:c02-BP2}. This means that $\delta_{j}^{L}$ is likely to get small if the neuron is near saturation. And this, in turn, means that any weights input to a saturated neuron will learn slowly\footnote{This reasoning won't hold if ${w^{l+1}}^T\delta^{l+1}$ has large enough entries to compensate for the smallness of $\sigma^\prime(z^l_j)$. But I'm speaking of the general tendency.}.

Summing up, we've learnt that a weight will learn slowly if either the input neuron is low-activation, or if the output neuron has saturated, i.e., is either high- or low-activation. 

None of these observations is too greatly surprising. Still, they help improve our mental model of what's going on as a neural network learns. Furthermore, we can turn this type of reasoning around. The four fundamental equations turn out to hold for any activation function, not just the standard sigmoid function (that's because, as we'll see in a moment, the proofs don't use any special properties of $\sigma$). And so we can use these equations to \textit{design} activation functions which have particular desired learning properties. As an example to give you the idea, suppose we were to choose a (non-sigmoid) activation function $\sigma$ so that  $\sigma^\prime$ is always positive, and never gets close to zero. That would prevent the slow-down of learning that occurs when ordinary sigmoid neurons saturate. Later in the book we'll see examples where this kind of modification is made to the activation function. Keeping the four Equations~\ref{eq:c02-BP1} to \ref{eq:c02-BP4} in mind can help explain why such modifications are tried, and what impact they can have.

{\centering \includegraphics[width=\textwidth,]{pic/tikz21.png} \par}

\textbf{Problem}

\begin{enumerate}
\item 
\textbf{Alternate presentation of the equations of backpropagation: }
I've stated the equations of backpropagation (notably Equations~\ref{eq:c02-BP1} and \ref{eq:c02-BP2}) using the Hadamard product. This presentation may be disconcerting if you're unused to the Hadamard product. There's an alternative approach, based on conventional matrix multiplication, which some readers may find enlightening. 
\begin{enumerate}
\item 
Show that Equation~\ref{eq:c02-BP1} may be rewritten as 
\begin{equation}
\delta^{L}=\Sigma^{\prime}\left(z^{L}\right) \nabla_{a} C
\label{eq:c02-33}
\end{equation}
where $\Sigma^{\prime}\left(z^{L}\right)$ is a square matrix whose diagonal entries are the values $\sigma^\prime(z^L_j$), and whose off-diagonal entries are zero. Note that this matrix acts on $\nabla_{a} C$ by conventional matrix multiplication. 

\item 
Show that Equation~\ref{eq:c02-BP2} may be rewritten as 
\begin{equation}
\delta^{l}=\Sigma^{\prime}\left(z^{l}\right)\left(w^{l+1}\right)^{T} \delta^{l+1}
\label{eq:c02-34}
\end{equation}
\item 
By combining observations (a) and (b) show that 
\begin{equation}
\delta^{l}=\Sigma^{\prime}\left(z^{l}\right)\left(w^{l+1}\right)^{T} \ldots \Sigma^{\prime}\left(z^{L-1}\right)\left(w^{L}\right)^{T} \Sigma^{\prime}\left(z^{L}\right) \nabla_{a} C
\label{eq:c02-35}
\end{equation}
\end{enumerate}

\item 
For readers comfortable with matrix multiplication this equation may be easier to understand than Equation~\ref{eq:c02-BP1} and Equation~\ref{eq:c02-BP2}. The reason I've focused on Equation~\ref{eq:c02-BP1} and Equation~\ref{eq:c02-BP2} is because that approach turns out to be faster to implement numerically. 
\end{enumerate}



\section{Proof of the four fundamental equations (optional)}
\label{sec:Proofofthefourfundamentalequations(optional)}

We'll now prove the four fundamental  Equations~\ref{eq:c02-BP1} to \ref{eq:c02-BP4}. All four are consequences of the chain rule from multivariable calculus. If you're comfortable with the chain rule, then I strongly encourage you to attempt the derivation yourself before reading on.

Let's begin with Equation~\ref{eq:c02-BP1}, which gives an expression for the output error,  $\delta^{l}$. To prove this equation, recall that by definition
\begin{equation}
\delta_{j}^{L}=\frac{\partial C}{\partial z_{j}^{L}}
\label{eq:c02-36}
\end{equation}
Applying the chain rule, we can re-express the partial derivative above in terms of partial derivatives with respect to the output activations, 
\begin{equation}
\delta_{j}^{L}=\sum_{k} \frac{\partial C}{\partial a_{k}^{L}} \frac{\partial a_{k}^{L}}{\partial z_{j}^{L}}
\label{eq:c02-37}
\end{equation}
where the sum is over all neurons $k$ in the output layer. Of course, the output activation $a^L_k$ of the \mth{k} neuron depends only on the weighted input $z_{j}^{L}$ for the \mth{j} neuron when $k=j$. And so $\partial a_{k}^{L} / \partial z_{j}^{L}$ vanishes when $k\neq j$. As a result we can simplify the previous equation to 
\begin{equation}
\delta_{j}^{L}=\frac{\partial C}{\partial a_{j}^{L}} \frac{\partial a_{j}^{L}}{\partial z_{j}^{L}}
\label{eq:c02-38}
\end{equation}



Recalling that $a_{j}^{L}=\sigma\left(z_{j}^{L}\right)$ the second term on the right can be written as $\sigma^{\prime}\left(z_{j}^{L}\right)$, and the equation becomes 
\begin{equation}
\delta_{j}^{L}=\frac{\partial C}{\partial a_{j}^{L}} \sigma^{\prime}\left(z_{j}^{L}\right)
\label{eq:c02-39}
\end{equation}
which is just Equation~\ref{eq:c02-BP1}, in component form.

Next, we'll prove Equation~\ref{eq:c02-BP2}, which gives an equation for the error $\delta^{l}$
in terms of the error in the next layer, $\delta^{l+1}$. To do this, we want to rewrite $\delta_{j}^{l}=\partial C / \partial z_{j}^{l}$ in terms of $\delta_{k}^{l+1}=\partial C / \partial z_{k}^{l+1}$. We can do this using the chain rule, 
\begin{eqnarray}
\delta_{j}^{l} &=&\frac{\partial C}{\partial z_{j}^{l}} \label{eq:c02-40}\\ 
&=&\sum_{k} \frac{\partial C}{\partial z_{k}^{l+1}} \frac{\partial z_{k}^{l+1}}{\partial z_{j}^{l}} \label{eq:c02-41}\\ 
&=&\sum_{k} \frac{\partial z_{k}^{l+1}}{\partial z_{j}^{l}} \delta_{k}^{l+1} \label{eq:c02-42}
\end{eqnarray}
where in the last line we have interchanged the two terms on the right-hand side, and substituted the definition of $\delta_{k}^{l+1}$. To evaluate the first term on the last line, note that 
\begin{equation}
z_{k}^{l+1}=\sum_{j} w_{k j}^{l+1} a_{j}^{l}+b_{k}^{l+1}=\sum_{j} w_{k j}^{l+1} \sigma\left(z_{j}^{l}\right)+b_{k}^{l+1}
\label{eq:c02-43}
\end{equation}


Differentiating, we obtain 
\begin{equation}
\frac{\partial z_{k}^{l+1}}{\partial z_{j}^{l}}=w_{k j}^{l+1} \sigma^{\prime}\left(z_{j}^{l}\right)
\label{eq:c02-44}
\end{equation}
Substituting back into Equation~\ref{eq:c02-42} we obtain 
\begin{equation}
\delta_{j}^{l}=\sum_{k} w_{k j}^{l+1} \delta_{k}^{l+1} \sigma^{\prime}\left(z_{j}^{l}\right)
\label{eq:c02-45}
\end{equation}
This is just Equation~\ref{eq:c02-BP2} written in component form.

The final two equations we want to prove are Equation~\ref{eq:c02-BP3} and Equation~\ref{eq:c02-BP4}. These also follow from the chain rule, in a manner similar to the proofs of the two equations above. I leave them to you as an exercise. 

\textbf{Exercise}
\begin{enumerate}
\item 
Prove Equation~\ref{eq:c02-BP3} and Equation~\ref{eq:c02-BP4}. 
\end{enumerate}

That completes the proof of the four fundamental equations of backpropagation. The proof may seem complicated. But it's really just the outcome of carefully applying the chain rule. A little less succinctly, we can think of backpropagation as a way of computing the gradient of the cost function by systematically applying the chain rule from multi-variable calculus. That's all there really is to backpropagation - the rest is details.

\section{The backpropagation algorithm}
\label{sec:Thebackpropagationalgorithm}


The backpropagation equations provide us with a way of computing the gradient of the cost function. Let's explicitly write this out in the form of an algorithm: 
\begin{enumerate}
\item 
\textbf{Input $x$:} Set the corresponding activation $a^1$  for the input layer.

\item \textbf{Feedforward:} For each $l=2,3,\ldots,L$ compute $z^{l}=w^{l} a^{l-1}+b^{l}$ and $a^{l}=\sigma\left(z^{l}\right)$.

\item \textbf{Output error  $\delta^{L}$:} compute the vector $\delta^{L}=\nabla_{a} C \odot \sigma^{\prime}\left(z^{L}\right)$.

\item \textbf{Backpropagate the error:} For each $l=L-1,L-2,\ldots,2$ compute\\ $\delta^{l}=\left(\left(w^{l+1}\right)^{T} \delta^{l+1}\right) \odot \sigma^{\prime}\left(z^{l}\right)$.

\item \textbf{Output:} The gradient of the cost function is given by\\ $\frac{\partial C}{\partial w_{j k}^{l}}=a_{k}^{l-1} \delta_{j}^{l}$ and $\frac{\partial C}{\partial b_{j}^{l}}=\delta_{j}^{l}$.

\end{enumerate}
 

Examining the algorithm you can see why it's called \textit{back}propagation. We compute the error vectors $\delta^{l}$
backward, starting from the final layer. It may seem peculiar that we're going through the network backward. But if you think about the proof of backpropagation, the backward movement is a consequence of the fact that the cost is a function of outputs from the network. To understand how the cost varies with earlier weights and biases we need to repeatedly apply the chain rule, working backward through the layers to obtain usable expressions.


\textbf{Exercises}
\begin{enumerate}
\item
\textbf{Backpropagation with a single modified neuron}\\
 Suppose we modify a single neuron in a feedforward network so that the output from the neuron is given by $f\left(\sum_{j} w_{j} x_{j}+b\right)$, where $f$  is some function other than the sigmoid. How should we modify the backpropagation algorithm in this case?
\item
\textbf{Backpropagation with linear neurons}\\
Suppose we replace the usual non-linear $\sigma$ function with $\sigma(z)=z$  throughout the network. Rewrite the backpropagation algorithm for this case. 
\end{enumerate}

As I've described it above, the backpropagation algorithm computes the gradient of the cost function for a single training example, $C=C_x$. In practice, it's common to combine backpropagation with a learning algorithm such as stochastic gradient descent, in which we compute the gradient for many training examples. In particular, given a mini-batch of $m$ training examples, the following algorithm applies a gradient descent learning step based on that mini-batch: 
\begin{enumerate}
\item
\textbf{Input a set of training examples}.
 
\item
\textbf{For each training example $x$}:
Set the corresponding input activation $a^{x,1}$, and perform the following steps:
\begin{enumerate}
\item
\textbf{Feedforward:} For each $l=2,3,\ldots,L$ compute $z^{x, l}=w^{l} a^{x, l-1}+b^{l}$ and $a^{x, l}=\sigma\left(z^{x, l}\right)$.

\item
\textbf{Output error $\delta^{x,L}$:} Compute the vector $\delta^{x, L}=\nabla_{a} C_{x} \odot \sigma^{\prime}\left(z^{x, L}\right)$.
\item

\textbf{Backpropagate the error:} For each $l=L-1,L-2,\ldots,2$ compute $\delta^{x, l}=\left(\left(w^{l+1}\right)^{T} \delta^{x, l+1}\right) \odot \sigma^{\prime}\left(z^{x, l}\right)$.

\end{enumerate} 

\item
\textbf{Gradient descent:} For each $l=L,L-1,\ldots,2$ update the weights according to the rule $w^{l} \rightarrow w^{l}-\frac{\eta}{m} \sum_{x} \delta^{x, l}\left(a^{x, l-1}\right)^{T}$, and the biases according to the rule $b^{l} \rightarrow b^{l}-\frac{\eta}{m} \sum_{x} \delta^{x, l}$.

\end{enumerate}

Of course, to implement stochastic gradient descent in practice you also need an outer loop generating mini-batches of training examples, and an outer loop stepping through multiple epochs of training. I've omitted those for simplicity. 

\section{The code for backpropagation}
\label{sec:Thecodeforbackpropagation}

Having understood backpropagation in the abstract, we can now understand the code used in the last chapter to implement backpropagation. Recall from that chapter that the code was contained in the \lstinline{update_mini_batch} and backprop methods of the \lstinline{Network} class. The code for these methods is a direct translation of the algorithm described above. In particular, the \lstinline{update_mini_batch} method updates the Network's weights and biases by computing the gradient for the current \lstinline{mini_batch} of training examples: 


\begin{fullwidth}
\begin{lstlisting}[language=Python]
class Network(object):
...
    def update_mini_batch(self, mini_batch, eta):
        """Update the network's weights and biases by applying
        gradient descent using backpropagation to a single mini batch.
        The ``mini_batch`` is a list of tuples ``(x, y)``, and ``eta``
        is the learning rate."""
        nabla_b = [np.zeros(b.shape) for b in self.biases]
        nabla_w = [np.zeros(w.shape) for w in self.weights]
        for x, y in mini_batch:
            delta_nabla_b, delta_nabla_w = self.backprop(x, y)
            nabla_b = [nb+dnb for nb, dnb in zip(nabla_b, delta_nabla_b)]
            nabla_w = [nw+dnw for nw, dnw in zip(nabla_w, delta_nabla_w)]
        self.weights = [w-(eta/len(mini_batch))*nw
                        for w, nw in zip(self.weights, nabla_w)]
        self.biases = [b-(eta/len(mini_batch))*nb
                       for b, nb in zip(self.biases, nabla_b)]
\end{lstlisting}
\end{fullwidth}


Most of the work is done by the line 
\begin{lstlisting}[language=Python]
delta_nabla_b, delta_nabla_w = self.backprop(x, y)
\end{lstlisting}
which uses the backprop method to figure out the partial derivatives $\partial C_{x} / \partial b_{j}^{l}$ and $\partial C_{x} / \partial w_{j k}^{l}$. The \lstinline{backprop} method follows the algorithm in the last section closely. There is one small change - we use a slightly different approach to indexing the layers. This change is made to take advantage of a feature of Python, namely the use of negative list indices to count backward from the end of a list, so, e.g., \lstinline{l[-3]} is the third last entry in a list \lstinline{l}. The code for \lstinline{backprop} is below, together with a few helper functions, which are used to compute the $\sigma$ function, the derivative $\sigma^\prime$, and the derivative of the cost function. With these inclusions you should be able to understand the code in a self-contained way. If something's tripping you up, you may find it helpful to consult the original description in Section~\ref{sec:Implementingournetworktoclassifydigits}.


\begin{fullwidth}
\begin{lstlisting}
class Network(object):
...

    def backprop(self, x, y):
        """Return a tuple ``(nabla_b, nabla_w)`` representing the
        gradient for the cost function C_x.  ``nabla_b`` and
        ``nabla_w`` are layer-by-layer lists of numpy arrays, similar
        to ``self.biases`` and ``self.weights``."""
        nabla_b = [np.zeros(b.shape) for b in self.biases]
        nabla_w = [np.zeros(w.shape) for w in self.weights]
        # feedforward
        activation = x
        activations = [x] # list to store all the activations, layer by layer
        zs = [] # list to store all the z vectors, layer by layer
        for b, w in zip(self.biases, self.weights):
            z = np.dot(w, activation)+b
            zs.append(z)
            activation = sigmoid(z)
            activations.append(activation)
        # backward pass
        delta = self.cost_derivative(activations[-1], y) * \
            sigmoid_prime(zs[-1])
        nabla_b[-1] = delta
        nabla_w[-1] = np.dot(delta, activations[-2].transpose())
        # Note that the variable l in the loop below is used a little
        # differently to the notation in Chapter 2 of the book.  Here,
        # l = 1 means the last layer of neurons, l = 2 is the
        # second-last layer, and so on.  It's a renumbering of the
        # scheme in the book, used here to take advantage of the fact
        # that Python can use negative indices in lists.
        for l in range(2, self.num_layers):
            z = zs[-l]
            sp = sigmoid_prime(z)
            delta = np.dot(self.weights[-l+1].transpose(), delta) * sp
            nabla_b[-l] = delta
            nabla_w[-l] = np.dot(delta, activations[-l-1].transpose())
        return (nabla_b, nabla_w)

...

    def cost_derivative(self, output_activations, y):
        """Return the vector of partial derivatives \partial C_x /
        \partial a for the output activations."""
        return (output_activations-y)

#### Miscellaneous functions
def sigmoid(z):
    """The sigmoid function."""
    return 1.0/(1.0+np.exp(-z))

def sigmoid_prime(z):
    """Derivative of the sigmoid function."""
    return sigmoid(z)*(1-sigmoid(z))

\end{lstlisting}
\end{fullwidth}

\textbf{Problem}
\begin{enumerate}
\item
\textbf{Fully matrix-based approach to backpropagation over a mini-batch:} Our implementation of stochastic gradient descent loops over training examples in a mini-batch. It's possible to modify the backpropagation algorithm so that it computes the gradients for all training examples in a mini-batch simultaneously. The idea is that instead of beginning with a single input vector, $x$, we can begin with a matrix $X=\left[x_{1} x_{2} \ldots x_{m}\right]$ whose columns are the vectors in the mini-batch. We forward-propagate by multiplying by the weight matrices, adding a suitable matrix for the bias terms, and applying the sigmoid function everywhere. We backpropagate along similar lines. Explicitly write out pseudocode for this approach to the backpropagation algorithm. Modify \lstinline{network.py} so that it uses this fully matrix-based approach. The advantage of this approach is that it takes full advantage of modern libraries for linear algebra. As a result it can be quite a bit faster than looping over the mini-batch. (On my laptop, for example, the speedup is about a factor of two when run on MNIST classification problems like those we considered in the last chapter.) In practice, all serious libraries for backpropagation use this fully matrix-based approach or some variant. 
\end{enumerate}

\section{In what sense is backpropagation a fast algorithm?}

In what sense is backpropagation a fast algorithm? To answer this question, let's consider another approach to computing the gradient. Imagine it's the early days of neural networks research. Maybe it's the 1950s or 1960s, and you're the first person in the world to think of using gradient descent to learn! But to make the idea work you need a way of computing the gradient of the cost function. You think back to your knowledge of calculus, and decide to see if you can use the chain rule to compute the gradient. But after playing around a bit, the algebra looks complicated, and you get discouraged. So you try to find another approach. You decide to regard the cost as a function of the weights $C=C(w)$ alone (we'll get back to the biases in a moment). You number the weights $w_1$,$w_2$,$\ldots$, and want to compute $\partial C / \partial w_{j}$ for some particular weight $w_j$. An obvious way of doing that is to use the approximation 
\begin{equation}
\frac{\partial C}{\partial w_{j}} \approx \frac{C\left(w+\epsilon e_{j}\right)-C(w)}{\epsilon}
\label{eq:c02-46}
\end{equation}
where $\epsilon>0$ is a small positive number, and $e_j$ is the unit vector in the \mth{j} direction. In other words, we can estimate $\partial C / \partial w_{j}$ by computing the cost $C$ for two slightly different values of $w_j$, and then applying Equation~\ref{eq:c02-46}. The same idea will let us compute the partial derivatives $\partial C / \partial b$ with respect to the biases.

This approach looks very promising. It's simple conceptually, and extremely easy to implement, using just a few lines of code. Certainly, it looks much more promising than the idea of using the chain rule to compute the gradient!

Unfortunately, while this approach appears promising, when you implement the code it turns out to be extremely slow. To understand why, imagine we have a million weights in our network. Then for each distinct weight $w_j$ we need to compute $C\left(w+\epsilon e_{j}\right)$ in order to compute $\partial C / \partial w_{j}$. That means that to compute the gradient we need to compute the cost function a million different times, requiring a million forward passes through the network (per training example). We need to compute $C(w)$ as well, so that's a total of a million and one passes through the network.

What's clever about backpropagation is that it enables us to simultaneously compute \textit{all} the partial derivatives $\partial C / \partial w_{j}$ using just one forward pass through the network, followed by one backward pass through the network. Roughly speaking, the computational cost of the backward pass is about the same as the forward pass\footnote{This should be plausible, but it requires some analysis to make a careful statement. It's plausible because the dominant computational cost in the forward pass is multiplying by the weight matrices, while in the backward pass it's multiplying by the transposes of the weight matrices. These operations obviously have similar computational cost.}. And so the total cost of backpropagation is roughly the same as making just two forward passes through the network. Compare that to the million and one forward passes we needed for the approach based on Equation~\ref{eq:c02-46}! And so even though backpropagation appears superficially more complex than the approach based on Equation~\ref{eq:c02-46}, it's actually much, much faster.

This speedup was first fully appreciated in 1986, and it greatly expanded the range of problems that neural networks could solve. That, in turn, caused a rush of people using neural networks. Of course, backpropagation is not a panacea. Even in the late 1980s people ran up against limits, especially when attempting to use backpropagation to train deep neural networks, i.e., networks with many hidden layers. Later in the book we'll see how modern computers and some clever new ideas now make it possible to use backpropagation to train such deep neural networks.

\section{Backpropagation: the big picture}
\label{sec:Backpropagation:thebigpicture}

As I've explained it, backpropagation presents two mysteries. First, what's the algorithm really doing? We've developed a picture of the error being backpropagated from the output. But can we go any deeper, and build up more intuition about what is going on when we do all these matrix and vector multiplications? The second mystery is how someone could ever have discovered backpropagation in the first place? It's one thing to follow the steps in an algorithm, or even to follow the proof that the algorithm works. But that doesn't mean you understand the problem so well that you could have discovered the algorithm in the first place. Is there a plausible line of reasoning that could have led you to discover the backpropagation algorithm? In this section I'll address both these mysteries.

To improve our intuition about what the algorithm is doing, let's imagine that we've made a small change $\Delta w_{j k}^{l}$ to some weight in the network, $w^l_{jk}$: 

\begin{figure}
\includegraphics{chapter02-06}
\end{figure}

That change in weight will cause a change in the output activation from the corresponding neuron: 

\begin{figure}
\includegraphics{chapter02-07}
\end{figure}

That, in turn, will cause a change in \textit{all} the activations in the next layer: 

\begin{figure}
\includegraphics{chapter02-08}
\end{figure}

Those changes will in turn cause changes in the next layer, and then the next, and so on all the way through to causing a change in the final layer, and then in the cost function: 

\begin{figure}
\includegraphics{chapter02-09}
\end{figure}


The change $\Delta C$ in the cost is related to the change ?$w^l_{jk}$ in the weight by the equation 
\begin{equation}
\Delta C \approx \frac{\partial C}{\partial w_{j k}^{l}} \Delta w_{j k}^{l}
\label{eq:c02-47}
\end{equation}
This suggests that a possible approach to computing $\frac{\partial C}{\partial w_{j k}^{l}}$ is to carefully track how a small change in $w^l_{jk}$ propagates to cause a small change in $C$. If we can do that, being careful to express everything along the way in terms of easily computable quantities, then we should be able to compute $\partial C / \partial w_{j k}^{l}$.

Let's try to carry this out. The change $\Delta w^l_{jk}$ causes a small change $\Delta a^l_j$ in the activation of the \mth{j} neuron in the \mth{l} layer. This change is given by 
\begin{equation}
\Delta a_{j}^{l} \approx \frac{\partial a_{j}^{l}}{\partial w_{j k}^{l}} \Delta w_{j k}^{l}
\label{eq:c02-48}
\end{equation}
The change in activation ?$a^l_j$ will cause changes in all the activations in the next layer, i.e., the (l+1)th layer. We'll concentrate on the way just a single one of those activations is affected, say $a^{l+1}_q$,


\begin{figure}
\includegraphics{chapter02-10}
\end{figure}

In fact, it'll cause the following change: 
\begin{equation}
\Delta a_{q}^{l+1} \approx \frac{\partial a_{q}^{l+1}}{\partial a_{j}^{l}} \Delta a_{j}^{l}
\label{eq:c02-49}
\end{equation}
Substituting in the expression from Equation~\ref{eq:c02-48}, we get: 
\begin{equation}
\Delta a_{q}^{l+1} \approx \frac{\partial a_{q}^{l+1}}{\partial a_{j}^{l}} \frac{\partial a_{j}^{l}}{\partial w_{j k}^{l}} \Delta w_{j k}^{l}
\label{eq:c02-50}
\end{equation}
Of course, the change $\Delta a^{l+1}_q$ will, in turn, cause changes in the activations in the next layer. In fact, we can imagine a path all the way through the network from $w^l_{jk}$ to $C$, with each change in activation causing a change in the next activation, and, finally, a change in the cost at the output. If the path goes through activations $a_{j}^{l}, a_{q}^{l+1}, \ldots, a_{n}^{L-1}, a_{m}^{L}$ then the resulting expression is 
\begin{equation}
\Delta C \approx \frac{\partial C}{\partial a_{m}^{L}} \frac{\partial a_{m}^{L}}{\partial a_{n}^{L-1}} \frac{\partial a_{n}^{L-1}}{\partial a_{p}^{L-2}} \ldots \frac{\partial a_{q}^{l+1}}{\partial a_{j}^{l}} \frac{\partial a_{j}^{l}}{\partial w_{j k}^{l}} \Delta w_{j k}^{l}
\label{eq:c02-51}
\end{equation}


that is, we've picked up a $\partial a/\partial a$ type term for each additional neuron we've passed through, as well as the $\partial C / \partial a_{m}^{L}$ term at the end. This represents the change in $C$ due to changes in the activations along this particular path through the network. Of course, there's many paths by which a change in $w^l_{jk}$ can propagate to affect the cost, and we've been considering just a single path. To compute the total change in $C$ it is plausible that we should sum over all the possible paths between the weight and the final cost, i.e., 
\begin{equation}
\Delta C \approx \sum_{m n p . \ldots q} \frac{\partial C}{\partial a_{m}^{L}} \frac{\partial a_{m}^{L}}{\partial a_{n}^{L-1}} \frac{\partial a_{n}^{L-1}}{\partial a_{p}^{L-2}} \ldots \frac{\partial a_{q}^{l+1}}{\partial a_{j}^{l}} \frac{\partial a_{j}^{l}}{\partial w_{j k}^{l}} \Delta w_{j k}^{l}
\label{eq:c02-52}
\end{equation}
where we've summed over all possible choices for the intermediate neurons along the path. Comparing with Equation~\ref{eq:c02-47} we see that 
\begin{equation}
\frac{\partial C}{\partial w_{j k}^{l}}=\sum_{m n p . \ldots q} \frac{\partial C}{\partial a_{m}^{L}} \frac{\partial a_{m}^{L}}{\partial a_{n}^{L-1}} \frac{\partial a_{n}^{L-1}}{\partial a_{p}^{L-2}} \cdots \frac{\partial a_{q}^{l+1}}{\partial a_{j}^{l}} \frac{\partial a_{j}^{l}}{\partial w_{j k}^{l}}.
\label{eq:c02-53}
\end{equation}

Now, Equation~\ref{eq:c02-53} looks complicated. However, it has a nice intuitive interpretation. We're computing the rate of change of $C$ with respect to a weight in the network. What the equation tells us is that every edge between two neurons in the network is associated with a rate factor which is just the partial derivative of one neuron's activation with respect to the other neuron's activation. The edge from the first weight to the first neuron has a rate factor $\partial a_{j}^{l} / \partial w_{j k}^{l}$. The rate factor for a path is just the product of the rate factors along the path. And the total rate of change $\partial C / \partial w_{j k}^{l}$ is just the sum of the rate factors of all paths from the initial weight to the final cost. This procedure is illustrated here, for a single path: 

\begin{figure}
\includegraphics{chapter02-11}
\end{figure}

What I've been providing up to now is a heuristic argument, a way of thinking about what's going on when you perturb a weight in a network. Let me sketch out a line of thinking you could use to further develop this argument. First, you could derive explicit expressions for all the individual partial derivatives in Equation~\ref{eq:c02-53}. That's easy to do with a bit of calculus. Having done that, you could then try to figure out how to write all the sums over indices as matrix multiplications. This turns out to be tedious, and requires some persistence, but not extraordinary insight. After doing all this, and then simplifying as much as possible, what you discover is that you end up with exactly the backpropagation algorithm! And so you can think of the backpropagation algorithm as providing a way of computing the sum over the rate factor for all these paths. Or, to put it slightly differently, the backpropagation algorithm is a clever way of keeping track of small perturbations to the weights (and biases) as they propagate through the network, reach the output, and then affect the cost.

Now, I'm not going to work through all this here. It's messy and requires considerable care to work through all the details. If you're up for a challenge, you may enjoy attempting it. And even if not, I hope this line of thinking gives you some insight into what backpropagation is accomplishing.

What about the other mystery - how backpropagation could have been discovered in the first place? In fact, if you follow the approach I just sketched you will discover a proof of backpropagation. Unfortunately, the proof is quite a bit longer and more complicated than the one I described earlier in this chapter. So how was that short (but more mysterious) proof discovered? What you find when you write out all the details of the long proof is that, after the fact, there are several obvious simplifications staring you in the face. You make those simplifications, get a shorter proof, and write that out. And then several more obvious simplifications jump out at you. So you repeat again. The result after a few iterations is the proof we saw earlier\footnote{There is one clever step required. In Equation~\ref{eq:c02-53} the intermediate variables are activations like $a^{l+1}_q$. The clever idea is to switch to using weighted inputs, like $z^{l+1}_q$, as the intermediate variables. If you don't have this idea, and instead continue using the activations $a^{l+1}_q$, the proof you obtain turns out to be slightly more complex than the proof given earlier in the chapter.} - short, but somewhat obscure, because all the signposts to its construction have been removed! I am, of course, asking you to trust me on this, but there really is no great mystery to the origin of the earlier proof. It's just a lot of hard work simplifying the proof I've sketched in this section.
 %[[Done]] How the Backpropagation Algorithm Works 
\include{c03} %Improving the Way Neural Networks Learn 
\include{c04} %[Partly done] A Visual Proof That Neural Nets Can Compute Any Function 
\include{c05} %Why are deep neural networks hard to train?
\include{c06} %deep learning
\include{c07} %[Partly done] algorithm for intelligence?   

%\part{Study Notes: Nelis Willers}
% -*- TeX -*- -*- UK -*- -*- Soft -*-

\chapter{Biological Systems}



\section{Biological Nervous Systems (BNSs)}

Branislav Holl\"{a}nder \cite{Hollander2018} wrote the text in this section.

Biological nervous systems, such as those found in vertebrates, operate in a different way. Take a look at the diagram. It shows a neuron cell connecting to another neuron. The cell consists of a cell body, with dendrites acting as connecting wires for other neurons to connect to. In most cases, a neuron has one axon capable of transmitting electric currents actively to other connecting cells. The connections between neurons are established using synapses located at the end of the axon. 
\begin{marginfigure}
\includegraphics{biologicalneurons}
\end{marginfigure}

These synapses are responsible for a lot of the magic of computing and memory in the nervous system. To see how this works, let us first examine the response of a single neuron to an incoming signal from its dendrites\cite{Synaptidude2005}.
\begin{marginfigure}
\includegraphics{biologicalneurons02}
\end{marginfigure}
We can observe that the synaptic membrane of the neuron typically has a resting potential of $-$70mV. In the resting state the cell does not carry any signal. In order to transmit a signal the cell has to be stimulated by its input synapses. In our example, the voltage threshold required to elicit a response is $-$55mV. This means that the total sum of the input has to amount to $+$15mV. If each input neuron only contributes $+$5mV, we need at least 3 such inputs to be active at the same time to provoke a response in the neuron. The mechanism is very similar to that of the artificial neurons in that the inputs get summed up before passing through a non-linearity. Inputs that cause a positive change in the membrane potential are called excitatory. In contrast to ANNs, in BNSs we can also find inhibitory inputs causing a negative change in the potential. These act to prevent the firing of an action potential.

We see that if the input is weaker than $+$15mV, there is no signal at the output. However, above the threshold, a strong active signal in generated, called the action potential. The potential rises sharply and is carried through the axon to the output synapses. After a short amount of time, the potential declines again. For some time after the decline, the cell may not be stimulated again (absolute refractory period).

The sharp thresholding of the input is equivalent to the non-linearity present in ANNs (the threshold in BNSs corresponds to a Heaviside non-linearity, just like in the early perceptron). Note that the firing of the action potential is binary: either it occurs or it does not. There is nothing in-between. This is called the all-or-nothing principle. Therefore, BNSs carry binary signals. This is in contrast to ANNs which carry continuous signals. However, the binary signals in NNNs are changing over time, which makes up for the reduced expressive power of binary values.

Of course, in order to transfer information in its binary form, BNSs have to use some kind of binary encoding, just as computers do. The most popular theories are that BNSs use frequency-based encoding and/or temporal encoding. In frequency-based encoding, the average frequency of impulses encodes the strength of a signal. Therefore, we might say that BNSs use frequency modulation. Notice that the average frequency is important in this mechanism, not the arrival time of individual impulses. This form of signal transfer is not very effective because it ignores the small time differences between individual impulses that could in principle carry information. However, frequency encoding is very robust to noise. Noise is a big concern in a electrochemical systems such as the nervous system.

Temporal encoding is similar to frequency encoding, except that the arrival time of individual impulses does matter in this case. A number of studies (see e.g. this paper) suggest that temporal encoding may be used at various places in our nervous system, such as in sensory systems. Temporal coding is very sensitive to noise, but may be necessary if a large amount of information needs to be transferred quickly.

The learning process of the human brain is not yet completely understood. However, we can say almost with certainty that the brain doesn't use a global learning method such as gradient descent. Gradient descent as is used in ANNs requires backpropagation. There is no biological mechanism for errors to be backpropagated further than a single neuron. Another issue is the definition of the loss function: do we as humans have inherent loss functions for various tasks defined in our brains? This is not very likely.

On the other hand, we found strong evidence (thanks largely to Eric Kandel, a Nobel laureate) that the brain is learning by using local methods such as Hebbian learning or \ac{STDP}. These methods are similar to learning in Restricted Boltzmann Machines in that they strengthen connections between neurons that often transmit signals between each other. The foundational concept of Hebbian learning is best explained by the author himself:

\begin{quote}
    Let us assume that the persistence or repetition of a reverberatory activity (or ``trace'') tends to induce lasting cellular changes that add to its stability. […] When an axon of cell A is near enough to excite a cell B and repeatedly or persistently takes part in firing it, some growth process or metabolic change takes place in one or both cells such that A's efficiency, as one of the cells firing B, is increased. (Hebb, D.O. (1949). The Organization of Behavior. New York: Wiley \& Sons.)
\end{quote}

STDP further enhances the Hebbian learning concept by making it time-asymmetric. More specifically, STDP strengthens the connection between neurons if input spikes in the first neuron occur immediately before output spikes in the second neuron. Otherwise, if the input spike of the first neuron occurs immediately after the output spike of the second neuron, the connection is weakened.

Hebbian learning is probably the most important learning mechanism occurring in the human brain. However, there is evidence that long-term memory may also work by regulating the presence/absence of connections between individual neurons, thus changing the very structure of the network. 

Despite of the current success of ANNs in the field of AI and Machine Learning, BNSs still exhibit multiple advantages, the most important being their extremely low power consumption. Current consensus among researchers suggests that the brain consumes approx. 20~W of energy. Compare that to an infinitely less capable Nvidia GPU consuming 300~W running a deep neural network. The low power consumption in turn requires much less cooling and allows the brain to build 3-D structures as opposed to electronic chips mostly following of 2-D designs.

Multiple research groups are currently trying to emulate some aspects of the nervous system \textit{in silico}. Such systems are called neuromorphic. Advantages of such systems should include, among others, high power efficiency, inherent massive parallelism and potentially real-time performance in inference.  % biological systems
% -*- TeX -*- -*- UK -*- -*- Soft -*-

\chapter{Visualizing Neural Nets}
\label{sec:VisualizingNeuralNets}

Prepared by CJ Willers.


\section{Overview}
\label{sec:OverviewAppc}

This document reports on analysis and experimental work to better understand neural nets, by visualizing the building blocks separately.  It is shown that a single neuron provides the same capability as a simple linear classifier; that is, to separate a data set into two classes, based on a linear discrimination surface (i.e., a straight line in the two-dimensional plane).  The mathematical shape of  the non-linear output of a solitary neuron is of little practical significance.

The almost magical power of neurons only really come to fruition when the neurons are connected in a network, when: 
\begin{enumerate}
\item more than one layer of neurons are used, 
\item the outputs of the neurons are limited in a non-linear manner, and 
\item the weights of all neurons in the network can be adjusted to optimise performance.
\end{enumerate}

When layers of non-linear neurons are combined, the simple linear discrimination capability of a single neuron can be applied to define complex discrimination curves.  These complex curves are constructed of smaller sections of the individual neuron discrimination lines.

\section{A Single Neuron}

The artificial neuron is the core computing element, simulating a biological neuron.  The artificial neuron receives a number of signals (small voltages) of arbitrary (analogue) value from other artificial neurons.  The artificial neuron then weights (multiplies by a set of values) the input signals, adds them together.  The sum is then distorted by a non-linear process, to yield the output signal. The output signal is then fed into neurons in the network.   The artificial neuron is modelled in mathematical form as shown in Figure~\ref{fig:neuralnetconcept}.

\begin{figure}[tb]
\centering
\includegraphics[width=\textwidth]{pic/neuralnetconcept}
\caption{Neural net concept}\label{fig:neuralnetconcept}
\end{figure}

Consider a very simple neuron with only two inputs $(x_1,x_2)$.  Then $\bm{x}=(x_1,x_2)$  and  $\bm{w}=(w_1,w_2)$, and the artificial neuron operation is $y = \varphi(\bm{x}\cdot\bm{w}+w_b)$, where $\varphi$  is any non-linear function and  $w_b$ is the neuron's bias weight.  Compare the mathematical expression of a two-input neural net with a simple straight line in a two-dimensional Cartesian space:

Neural net: $z=\varphi(w_1x_1+w_2x_2+b)$

Straight line:  $0=(mx+(-1)y+c)$

Note the remarkable resemblance between the two equations.  Suppose, for a short while, that the non-linear function can be 'forced' to be linear (in other words, the $\varphi()$ function   disappears).  Also consider the neuron when $z=0$.  Then the equation for the straight line and the (assumed simplified) neuron have the same form.  Under these two conditions, one can state: $w_1=m$, $w_2=-1$, $x_1=x$, $x_2=y$, $b=c$.  It is evident that the two weights and bias of a two-input neuron defines a straight line.  Returning to fix the two simplifying assumptions, we note that the straight line now describes the locus (line) where the non-linear function $\varphi()$ is zero.  

Several nonlinear functions are used in neural nets, including the sigmoid (logistics), tansig, or soft max
functions (see Appendix~\ref{sec:ActivationFunctions}). 
For this discussion consider the tansig function, but the same also applies to the sigmoid. 
Imagine a sheet of paper formed into a two-dimensional shape (see Figures~\ref{fig:tansig}).  Visualize walking along the $z=0$ contour of the tansig surface.  As we walk along this line, to the left of us, the surface falls away to smaller values, while on the right hand side, the surface rises to larger values.  Note, however, that the tansig $z=0$ line is the equation of the straight line derived from the neuron equation, $0=z=\varphi(w_1x_1+w_2x_2+b)$ , since $\varphi(x)=0$.  The neuron's weights and bias therefore determines the exact location and orientation or direction of the tansig surface in the two-dimensional Cartesian space.  

Note that while $\varphi(w_1x_1+w_2x_2+b)$  and  $\varphi(2w_1x_1+2w_2x_2+2b)$ both result in the same straight line in the two-dimensional plane, they yield two very different tansig surfaces.  The slope of the surface, away from the  $\varphi(x)=0$ line, is strongly affected by the neuron's weights and the bias introduces a shift perpendicular to, and 'up/down' relative to the $z=0$ line, the $z=0$ line because of shifting the tansig  argument.  

\section{Linear Classifiers and Neurons}

\begin{figure}[tb]
\centering
\includegraphics[width=.6\textwidth]{pic/chC-trainingset}
\caption{Neural net concept}
\label{fig:chC-trainingset}
\end{figure}

Figure~\ref{fig:chC-trainingset} shows data from two classes plotted in terms of feature $x_1$ and feature $x_2$.  Objects from class O are marked with a circle, and objects from class X are marked with an asterisk.  

The task at hand is to determine a method to discriminate between the two classes O and X on the basis of the feature sets $x_1$ and $x_2$.  While it is easy to do this by visual observation, the objective is to determine an accurate and robust mathematical means to perform this task.  

As a first attempt, draw a straight line as shown in Figure~\ref{fig:chC-trainingset}.  This line is a simple linear (straight line) classifier, where the line $x_2=0.7x_1-1$  represents the classification boundary.  All objects with features 
'above' the line belongs to class X, while the object with features 
'below' the line belongs to class O.  Observing the figure carefully, one notices that not all the class X-objects fall 'above' the line.  There are also some class O objects above the line.   In other words, the two classes denoted by X and O cannot be discriminated by this simple discriminator.  

Consider next a 2-D tansig sheet superimposed around the linear classifier line.  This is done by orienting the tansig $z=0$ along the straight line defined by the linear classifier (bottom graphs in Figure~\ref{fig:tansig}).  This seemingly simple operation has significant implications.  We now have a two-dimensional \textit{(two-input) neuron as a linear discrimination element}, but with a non-linear amplitude output.    It is very important to observe that this two-input neuron defines a single, straight line between two classes, and that the classification 'line' has nothing to do with the non-linear scaling in the output - the discrimination ability of the single neuron is determined primarily by its weights and bias.  Sadly, still, this single neuron still has the same limitations as linear discriminator. 

Suppose we can add more straight lines (neurons), and somehow define in which regions they should operate --- then we can define a complex discrimination line as a set of shorter segments.  Enter neural networks.

\section{Neural networks}

Adding more discrimination lines is easy: just add more input neurons.  If we need N segments to define the discrimination line in the two-dimensional plane, use N neurons.   Note that, in effect, we are using N linear classifiers (straight lines), but we have to limit the operating range of each classifier to a small segment of its total length.    In effect, this means that all the straight lines are optimised to achieve the best discrimination, but each in his own (local) area.  This is where the hidden layer(s) come to the rescue:  each first-layer neuron's output is scaled by the weight where this neuron feeds into the second layer, thereby 'activating' only selected sections of each of the linear classifiers.  
The sections of the classifier that are not used, are pushed into upper or lower saturation by the nonlinear tansig function. The saturation, combined by the weight feeding into the next layer, relegate these sections to having no effect. 
Note, however, that the optimisation is not done separately for each line - all elements in the net are trained as a whole, with full consideration of the interaction and dependencies between elements.  


\begin{figure}[tb]
\centering
\includegraphics[width=0.8\textwidth]{pic/neuralnetconcept02}
\caption{Small neural net}\label{fig:neuralnetconcept02}
\end{figure}


To illustrate these concepts, a small neural net was trained to discriminate between the O and X classes in Figure~\ref{fig:chC-trainingset}.  Four hidden layer neurons and two output neurons were used (Figure~\ref{fig:neuralnetconcept02}).  There are only two inputs $(x_1,x_2)$ , feeding into the hidden layer neurons.  When the net is trained, the weights and bias values are adjusted to find an optimum discrimination.  Class X object provide an output signal of $-1$, while a Class O object provides a $+1$ output signal.  

In our journey through neural network land,  we start of by considering each neuron on its own, then the neural net is assembled by adding the hidden layer neurons one by one, until we reach the final network.  The three-dimensional graphs shown in Figure~\ref{fig:chC-hiddeneurons04} represent the output of the hidden neurons, for all possible input values  $(x_1,x_2)$  with $0\leq x_1 \leq 10$  and  $0\leq x_2 \leq 10$.    For each of the $(x_1,x_2)$ points, the neuron responses were calculated and plotted in the three-dimensional graph format.   We can see how the neuron will respond to all possible input signal combinations.   By careful study of these input-output graphs, the inner workings of the neural net become clear.
 
\begin{figure}[tbhp]
\centering
\includegraphics[height=1.1\textwidth,angle=90]{pic/chC-hiddeneurons04}
\caption{Small neural net neuron responses}
\label{fig:chC-hiddeneurons04}
\end{figure}

First, we will study the tansig surface shape of each neuron hidden neuron, and show how these hidden neurons each contribute a portion to the overall discrimination line.  Note that not all neurons seem to have the same dramatic effect, but keep in mind that the next layer can scale up this output when adding it to the other neuron signals.  In fact, the key to the neural network calculation is the fact that the outputs from these hidden layer neurons must be added in exactly the correct proportions to achieve the final discrimination objective.  
 
The graphs in Figure~\ref{fig:chC-hiddeneurons04} show the individual responses of the hidden neurons.   The top row shows that each of the neurons only provide a simple straight line discrimination.   The lines are widely different in slope and offset.  The outputs of these hidden neurons are also contained between $-1$ and $+1$, by the tansig function.   
The graph on the bottom right in Figure~\ref{fig:chC-classif04} shows the $y=mx+c$ lines for each of the hidden neurons.  Each line in this graph  traces out the $z=0$  for one of the tansig surfaces.  Note how each of the lines attempt to provide a measure of discrimination between the O and X classes.  None of these lines are able to provide the required separation between O and X classes.   

The second and third rows in Figure~\ref{fig:chC-hiddeneurons04} show the sum of the neuron signals, starting with the bias, and then adding an additional neuron until all neuron signals are included.  The signals shown here are already scaled by the output neuron weights, i.e., it is the output neuron activation, but the non-linear output compression is not yet applied. 
Note how the output neuron weights (sign and magnitude) manipulate the signals from each of the previous layer neurons to build a surface that will eventually form the discrimination surface.  In particular, note how the class X area is growing in a positive direction, while the class O area is growing towards the negative.  Keep in mind that the class X output must be $+1$ while the class O output must be $-1$.  As the neuron signals are added, the emerging shape approaches the required discrimination class regions 


In experimenting with the net, it was quite interesting to observe how much the net behaviour changes, for just one small adjustment in the training set.  Even when using the same input data, different runs resulted in dramatically different shapes, just because of random differences in starting weights and biases.  The results shown here, therefore only demonstrate one possible solution, there are many other (equally valid) solutions.  These other solutions are obtained by repeated training, each time with different seeds to the random number generator. 



\FloatBarrier
 
\begin{figure}[p]
\centering
\includegraphics[width=\textwidth]{pic/chC-classifa04}
\includegraphics[width=\textwidth]{pic/chC-classifb04}
\caption{Small neural net classifier, four hidden neurons, seed value 10}
\label{fig:chC-classif04}
\end{figure}

 
\begin{figure}[p]
\centering
\includegraphics[width=\textwidth]{pic/chC-classifa08}
\includegraphics[width=\textwidth]{pic/chC-classifb08}
\caption{Small neural net classifier, eight hidden neurons, seed value 10}
\label{fig:chC-classif08}
\end{figure}
 
  
\begin{figure}[p]
\centering
\includegraphics[width=\textwidth]{pic/chC-classifa05}
\includegraphics[width=\textwidth]{pic/chC-classifb05}
\caption{Small neural net classifier, four hidden neurons, seed value 1}
\label{fig:chC-classif05}
\end{figure}

The output signal from the output neurons (before and after non-linear compression) are shown in Figure~\ref{fig:chC-classif04}.   It is quite informative to study the graph and to identify the individual linear discrimination lines, and to observe how these lines are cleverly segmented by adding the neurons' signals together.   It is also evident that the hidden neuron signals  were scaled differently by the input weights on the output neuron.  

The key observation is that the hidden neuron outputs are limited by the tansig signal, and hence, their output values can be scaled \textit{differently} by the weights in the output layer.  If the hidden layer outputs were not limited, the output signals would be  increasing/decreasing away from the $\varphi(0)=0$  line, and there would be little value in adding these increasing values together. 

This output response is the result of a happy dance of 16 weights, and 6 bias values, for the full range of input feature values.  Note that this output neuron activation signal (before the tansig compression) can vary wider than $-20$ to $+20$, depending on the input  $(x_1,x_2)$.    The distinctly visible plateaus in Figure~\ref{fig:chC-classif04} are remnants from the tansig shapes of the hidden neurons.  

Figure~\ref{fig:chC-classif08} shows a classifier using eight hidden neurons.  Despite using more hidden neurons, the classifier performs poorer than the four-neuron-classifier.  The figure shows that only three of the eight neurons actually contributes to the classifier; the other neurons' classifier lines are not even in the input values' domain.  Clearly more is not better than less.

Figure~\ref{fig:chC-classif05} shows a classifier also using four hidden neurons, but with a different initial weights and bias set, because the random number generator seed value is different. The results are significantly different from Figure~\ref{fig:chC-classif04}, but is not unlike the solution for the eight-hidden-neuron case in Figure~\ref{fig:chC-classif08}.  This case shows that experimentation with all the different model parameters is required to find the optimal solution.  The solution is not obtained by simple analytical process or by a 'first-time-lucky' experiment.


\section{Interpreting the Output}

We are  interested in classifying between class O and class X.  The classifier has two 

, we therefore define a threshold value of $0$, and compare the net output against this threshold to classify an object. 
\begin{enumerate}
\item where the net output is greater than zero is class X domain, and 
\item where the net output is less than zero is class O domain.  
\end{enumerate} 
the final shaping of the discrimination curve is shown on the left.  The two top graphs show the final output of the neural net before and after non-linear compression.  

The second graph shows the output of the neural net after the non-linear limiting in the output neuron.  Notice how effectively all the plateaus and valleys are removed, leaving only the discrimination curve.  Note, also the input neurons' straight line classification segments.

The third curve shows a plan or top view of the two-dimensional Cartesian space.   The original input values are shown, together with the neural net response, shown in a contour plot.  A contour plot shows constant value lines (exactly like contour lines on a map shown altitude).  The $z=0$  line is somewhere in the turquoise colour band (near the middle of the band).  

Everything to the left (blue) side of $z=0$  is class O,  while the space to the right (red) side of $z=0$  is class X.  Observe how effectively the net employs the 5 linear discriminators to build the complex discrimination shape.  

We can only visualize curves like these up to the third dimension, but it some means of visualization for higher dimensions are available, it could be a useful tool to confirm the net design.

\FloatBarrier
\section{Alternative Perspective}

This chapter provides compelling evidence to consider an artificial  neural net as a cascaded collection of linear discrimination building lines, planes or hyperplanes.  The graphs clearly show a number of lines and various plateaus and valleys that meet our intuitive notion of 'discrimination'.

Note, however, that neither the neural net, nor the back-propagation algorithm has any conscious knowledge of any such discrimination lines of surfaces.  In the same sense that a falling rock has no knowledge of Newton's laws, the net performs its function without being bound by our own poorly construed mental pictures of how things work.

An alternative view on the operation of the neural net, which is just as valid as the  view described here, it to consider the neural net as a very large (but conceptually simple) mathematical equation.  This equation is formed by a large number of sub-equations (neurons).  The end effect is a black box (the equation) that has a number of inputs and a number of outputs.  Inside this black box, are the equation itself, plus a large number of parameters (weights and biases) that must be adjusted to the problem at hand.  In terms of our two dimensional problem, this is the same as fitting a $y=mx+c$ curve to a number of data points: the slope $m$ and offset $c$ are parameters that must be adjusted to minimize the curve-fit error.  

In our little experiment above, the training of the neural net meant finding the best set of  linear discriminators.  However, in terms of the black box model, training the net means finding the set of parameters (weights and biases) that minimizes the output error for the given training set.  The output error is defined as the mean error from the ideal output, for all the neurons, taken collectively.  This error is normally defined as the mean value of the square of all the errors from all the neurons
\begin{equation}
e = \frac{1}{N}\sum^N_{n=1}(o_n-t_n)^2
\end{equation}
where $e$  is the mean squared error,  $N$  is the number of test vectors in the training set,  $o$  is the observed neural net output signal, and $t$  is the target or true value in the training data.  The purpose with the training process is to minimize $e$  by finding the best set of weights and biases.  

Seen as a black box, with adjustable parameters, it becomes imperative that optimal parameter estimation procedures be found.  The neural network fraternity has done much work in developing algorithms and tools in this regard.  Large numbers of optimisation algorithms have been developed for the variety of network topologies.  The simple feed forward topology considered here is only one example of the larger set of neural networks.

The black box model simply says that the underlying mathematical equation is used as the basis and the best parameters must be found to fit the data.  Once these parameters are known, the black box will perform according to these parameters.  The training procedure therefore strives to find the best fit parameters, relative to the training data set.  The fact that this is a neural network, is only of consideration to derive the underlying mathematical equation, it is of no consequence once the equation has been finalized.  

\section{Conclusion}

This study only covers two-dimensional space, where the discrimination boundary is a straight line.  In three-dimensional feature space (three inputs), the boundary is a plane, where the plane's location and orientation is defined by the three weights and the bias.  In higher order dimensions, the discrimination boundaries become hyper-planes.  Even though visualization becomes difficult, the fundamental principle still remains as outlined in this document for two dimensions.

The weights of the first layer neurons define the linear discriminators, while the weights in subsequent layers manipulates the linear discriminators so that only the required segments are effective.  On a meta level, one can say that the next layer weights perform linear discrimination on the outputs of the previous layer neurons.  This effect of abstract levels of next layers manipulating previous layers becomes hard to visualize.

It is generally accepted that the human brain has neuron connectivity of the order \num{10e4} (see \cite{WikiPediaNeuron2019}), this means that the brain can assemble discrimination hyper-surfaces in \num{10e4}  dimensional space.  Furthermore, the brain's complex feedback paths and ability to have thousands of 'hidden layers' hints that the brain can form discrimination surfaces of almost unlimited complexity.  Our work with small networks are but feeble attempts in comparison with the complexity of natural neural networks.  
 %visualising classifiers
\include{c10} %Activation Functions
% -*- TeX -*- -*- UK -*- -*- Soft -*-

\chapter{Data Preparation}
\label{sec:DataPreparation}

\TBC{To be completed}

Input data can be be widely varying, some form of normalisation is required to ``pull'' the inputs into the same range, so that the large values don't dominate the smaller values.  

One of the standard techniques is to scale each set of input data independently to a mean value of zero and a unit variance.
Scikit-Learn's \lstinline{StandardScaler} can do this for you.


The Ultimate Guide to 12 Dimensionality Reduction Techniques (with Python codes) \cite{PulkitSharma2018}.
1 Missing Value Ratio,
2 Low Variance Filter,
3 High Correlation Filter,
4 Random Forest,
5 Backward Feature Elimination,
6 Forward Feature Selection,
7 Factor Analysis,
8 Principal Component Analysis,
9 Independent Component Analysis,
10 Methods Based on Projections,
11 t-Distributed Stochastic Neighbor Embedding (t-SNE), and 
12 UMAP. %data preparation
\include{c12} %loss functions
\include{c13} %Regularization

%\part{Supplementary Notes}
% -*- TeX -*- -*- UK -*- -*- Soft -*-

\part{Supplementary Notes}

\chapter{Environment Setup}
\label{sec:EnvironmentSetup}

Prepared by CJ Willers.

\section{General Python Environment}

\TBC{Set up Anaconda}

 %Environment Setup
% -*- TeX -*- -*- UK -*- -*- Soft -*-

\chapter{Python and Jupyter Code}
\label{sec:PythonandJupyterCode}

Prepared by CJ Willers.


\section{Code Supplement to Chapter~4}
\label{sec:CodeSupplementtoChapter4}



The repository that contains the \LaTeX{} source code has a Jupyter notebook with interactive plotting widgets, see \lstinline{code/chapter4-graphics.ipynb}.

\subsection{Visualising Weight and Bias Effects}
\label{sec:VisualisingWeightandBiasEffects}

Equation~\ref{eq:c01-03-sigmoidfunction} and following code shows how the neuron weight and bias affects the sigmoid output.

\begin{lstlisting}
x = np.linspace(0,1,300)
p = ryplot.Plotter(1,4,4,figsize=(15,15),doWarning=False)
ws = [5,10,20,40]
bs = [-20,-10,-5,-2]
for iw,w in enumerate(ws):
    for ib,b in enumerate(bs):
        y = sigmoid(x,w,b )
        p.plot(1+ib+iw*len(bs), x, y, f'w={w} b={b}',maxNX=1,maxNY=1,xAxisFmt='%.0f',yAxisFmt='%.0f',pltaxis=[0,1,0,1] )
\end{lstlisting}

{\centering \includegraphics[width=\textwidth,]{sigmoid4x4plot} \par}

Observe how the output (vertical axis) changes as $w$ changes (downward in the set of graphs).


\subsection{Approximating Functions with Neural Nets}
\label{sec:ApproximatingFunctionswithNeuralNets}

The following Python code provides slider functions in a Jupyter notebook, providing similar functionality to the interactive graphics on the web site.  Open a Jupyter notebook or lab book and enter and play with the code.  See the code for comments.

\begin{lstlisting}[language=Python]
import numpy as np
import matplotlib.pyplot as plt

%matplotlib inline


def sigmoid(x,w,b):
    return 1 / (1 + np.exp(- x * w - b))
    

def plot_activation_sh(**kwargs):
    """Plot the activation of a single output neuron from a set of hidden neurons,
    given a list of s values (hidden neuron bias/weight)
    and a list of h values (output neuron weight for each hidden neuron).
    
    The list has a 1.5 x number of hidden neurons tuples.
    Each tuple defines one slider, in terms of its name, default value and lo/hi limits.
    
    Each pair of two hidden neurons (with different s-values) share one h-weight value.
    The two s-values define the lower and upper s-values, defining an interval on the x axis.
    The output of the two s-value neurons are subtracted and multiplied with the h-weight value.
    This set of three values (2xs and 1xh) defines one 'pulse' along the x-axis with height h.
    
    The network parameters are given as **kwargs, on the following understanding:
    1) tuples with s values (one tuple for each hidden neuron input weight, s=bias/weight)
    2) tuples with h values (one tuple shared between each pair of two hidden neurons)
    """
    numNeuronPairs = int(len(kwargs)/3)
    x = np.linspace(-1,1,300)
    wh = 300
    sum = 0
    for i in range(numNeuronPairs):
        lonum = f'{i*2+0:02d}'
        hinum = f'{i*2+1:02d}'
        sum += kwargs[f'h{lonum}-{hinum}'] * sigmoid(x,wh,-kwargs[f's{lonum}']*wh) - \
            kwargs[f'h{lonum}-{hinum}'] * sigmoid(x,wh,-kwargs[f's{hinum}']*wh) 
    svals = [kwargs[f's{s:02d}'] for s in range(2*numNeuronPairs)]
    fig, ax = plt.subplots(figsize=(8, 6))
    ax.set_xlabel('x')
    ax.set_ylabel('y')
    ax.plot(x, sum,  linewidth=2)
    ax.set_ylim(np.min(sum),np.max(sum))
    ax.set_xlim(min(svals),max(svals))

    
def setupShow_sh(ss,hs,hidden=False,step=0.01):
    """Load tuples and set up the sliders.
    
    ss: list of s-value tuples
    hs: list of h-value tuples
    hidden: if True don't display sliders
    step: slider step values
    
    """
    sliders = []
    for i,(lo,hi,val) in enumerate(ss):
        slid = widgets.FloatSlider(value=val,min=lo,max=hi,step=step,
                            description=f's{i:02d}',disabled=False,continuous_update=False,
                            orientation='horizontal',readout=True,readout_format='.2f')
        if hidden==True:
            slid.layout.display = 'none'
        sliders.append(slid)

    for i,(lo,hi,val) in enumerate(hs):
        slid = widgets.FloatSlider(value=val,min=lo,max=hi,step=step,
                            description=f'h{2*i:02d}-{2*i+1:02d}',disabled=False,
                            continuous_update=False,orientation='horizontal',
                            readout=True,readout_format='.2f')
        if hidden==True:
            slid.layout.display = 'none'
        sliders.append(slid)

    kwargs = {slider.description:slider for i,slider in enumerate(sliders)}
    interact(plot_activation_sh,**kwargs);

    
# ss is a list of slider s-values (min, max, value)
# hs is a list of slider h-values (min, max, value)
case = 1
if case==0:
    ss = [(-1, 1,-0.9),(-1, 1,-0.71), (-1, 1,0.01), (-1, 1,0.36)]
    hs = [(-2, 2, 0.69),(-2, 2, -1.52)]
    hidden = False
elif case==1:
    ss = [(-1, 1,0),(-1, 1,0.2), (-1, 1,0.2), (-1, 1,0.4), (-1, 1,0.4), (-1, 1,0.6), 
          (-1, 1,0.6), (-1, 1,0.8), (-1, 1,0.8), (-1, 1,1)]
    hs = [(-2, 2, 0.3),(-2, 2, -1.0),(-2, 2, 0.2),(-2, 2, -1.2),(-2, 2, -0.3)]
    hidden = True

setupShow_sh(ss,hs,hidden=hidden)    
\end{lstlisting}

The Jupyter notebook should display the sliders (if not hidden) and the reconstructed graph.

{\centering \includegraphics[width=\textwidth,]{wigglyfn21} \par}

 %Python and Jupyter Code
% -*- TeX -*- -*- UK -*- -*- Soft -*-

\chapter{Google Colaboratory}
\label{sec:Google Colaboratory}

The early content of chapter is based on material from  \cite{LiuCoLab2019,AnneBonner2019}.
\section{Background}

See the FAQ
\lstinline{https://research.google.com/colaboratory/faq.html}

It is an online  free research tool storing colab notebooks on Google Drive. Colab notebooks can import from Jupyter notebooks.

The code is executed in a virtual machine dedicated to your account. Virtual machines are recycled when idle for a while, and have a maximum lifetime enforced by the system. 

Colaboratory is intended for interactive use. Long-running background computations, particularly on GPUs, may be stopped. We encourage users who wish to run continuous or long-running computations through Colaboratory’s UI to use a local runtime, see \cite{googleLcalColabs2019}.

Colab is ideal for everything from improving your Python coding skills to working with deep learning libraries, like PyTorch, Keras, TensorFlow, and OpenCV. You can create notebooks in Colab, upload notebooks, store notebooks, share notebooks, mount your Google Drive and use whatever you’ve got stored in there, import most of your favorite directories, upload your personal Jupyter Notebooks, upload notebooks directly from GitHub, upload Kaggle files, download your notebooks, and do just about everything else that you might want to be able to do.

\section{Working with Colaboratory Notebooks}
\subsection{Getting Started}
\begin{marginfigure}
\includegraphics{drivecolab}
\end{marginfigure}
Start by making a folder in your Google Drive to store the colab notebooks and then create a new colab notebook. Note that you can also create a new notebook directly from colab itself.

Open \lstinline{https://colab.research.google.com} in a browser and click on \lstinline{NEW PYTHON 3 NOTEBOOK} in the lower right corner.  This should launch a notebook in your browser. 

Browse around and you will find lots of useful information on the welcome page.

Once you have a notebook created, it'll be saved in your Google Drive (Colab Notebooks folder). You can access it by visiting your Google Drive page, then either double-click on the file name, or right-click, and then choose ``Open with Colab''.  Alternatively you can open it from within colab from the menu by File$>$Open and then click on the Google Drive tab.

\begin{marginfigure}
\includegraphics{drivecolab05}
\end{marginfigure}

Rename the notebook by clicking on the notebook's name or use the File$>$Rename menu.

\subsection{Connecting to a Notebook}
After some time of inactivity or after the maximum session time, Google will terminate the session.
The session can be reconnected (or changed to a local or hosted runtime) by clicking in the drop-down box in the top right of the window. Select appropriately and after some time the session will connect and a green tick mark will appear showing how much RAM and disk space are currently in use. 
\begin{marginfigure}
\includegraphics{drivecolab06}
\end{marginfigure}


\subsection{Saving on GitHub}

Create a new repository, and commit at least one file to the master branch, so that the github repo has a master branch (not an empy repo).
In the colab File menu select save copy to github.
You will be asked to authorise the access and henceforth it should work without further authorisation.
Select the require repository and its master branch and then click on OK.
In future, the file can be opened from within GitHub by clicking on the button.

Just be careful if saving colab notebooks, it will overwrite files by the same name in GitHub.
\begin{figure*}[h]
\includegraphics[width=\textwidth]{colabrepo01}
\end{figure*}


\subsection{Enabling GPU Support}

To turn on GPU for your Deep Learning projects, just go to the drop-down menu and select Runtime$>$Change runtime type$>$Hardware accelerator and choose GPU

\subsection{Working with the Notebook}

The colab notebooks are similar to Jupyter notebooks, use the same commands.
You can also save files to your colab folder (and presumably then also to github).

\subsection{Mounting Google Drive}

If you want to mount your Google Drive for file access, execute

\begin{lstlisting}
from google.colab import drive
drive.mount('/content/gdrive')
\end{lstlisting}

Google will respond by requesting you to authorise access to your user account and then its Drive services, respond by approving. You will be given an authorisation code which must be copied and pasted in the box provided. Once done, you will be informed of the mount point.
\begin{marginfigure}
\includegraphics{drivecolab02}
\end{marginfigure}
The drive will also be visible in the Files tab on the left explorer pane.  You can list current folder's contents with the Unix 'ls' command.
\begin{marginfigure}
\includegraphics{drivecolab03}
\includegraphics{drivecolab04}
\end{marginfigure}

\subsection{Uploading and Downloading Files}

Files can be uploaded by clicking on the UPLOAD button in the explorer pane.


Use wget to download and if it is a zipped file, unzip to get the contents.

\begin{lstlisting}
!wget -cq https://s3.amazonaws.com/content.udacity-data.com/courses/nd188/flower_data.zip
!unzip -qq flower_data.zip
\end{lstlisting}

\subsection{Long Running Processes}

Colaboratory is intended for interactive use. Long-running background computations, particularly on GPUs, may be stopped. ... Google encourages users who wish to run continuous or long-running computations through Colaboratory’s UI to use a local runtime or Google's paid services.




\subsection{Terminating a Session}

Virtual machines are recycled when idle for a while, and have a maximum lifetime enforced by the system.
It's 90 minutes if you close the browser. 12 hours if you keep the browser open. Additionally, if you close your browser when a code cell is running, if that same cell has not finished, when you reopen the browser it will still be running (the current executing cell keeps running even after browser is closed)
When you close the browser while a cell is still running the results will not end up saved on Google Drive.


To see a list of all the open sessions or to terminate a session on the menu click on Runtime$>$Manage sessions.  A new window will open that shows all open sessions.  The same window allows the termination of a session.
\begin{marginfigure}
\includegraphics{drivecolab07}
\end{marginfigure}




\section{Working with Colab Notebooks}

Colab notebooks are very similar to Jupyter notebooks.
Run Python code in the same manner.

Use the ! prepend to execute system level commands such as \lstinline{!pip} or \lstinline{!apt-get install}:
\begin{lstlisting}
!pip install -q keras
import keras
\end{lstlisting}


\section{Introduction to TensorFlow}

\subsection{Tensors}

TensorFlow bases its name on the word ``tensor''. What is a tensor anyway? In short, a multi-dimensional array. Let's see what that means!

\begin{itemize}
\item We have one single number, e.g. 6, we call it a ``scalar'';
\item We have three numbers, e.g. [ 6, 8, 9], we call that a ``vector'';
\item We have a table of numbers, e.g. [[6, 8, 9], [2, 5, 7]], we call that a ``matrix'' (which has two rows and three columns);
\item We have a table of table of numbers, e.g. [[[6, 8, 9], [2, 5, 7]], [[6, 8, 9], [2, 5, 7]]], and…we are running out of words here :( My friend, that is a tensor! A tensor is just a generalized form of arrays that can have any number of dimensions.
\end{itemize}

In TensorFlow jargons, a scalar is a rank 0 tensor, a vector is rank 1 and matrix rank 2 etc. There are three frequently used types of tensors: constant, variable, and placeholder which are explained below.

\subsection{Types of Tensors}

Constants are exactly what their names refer to. They are the fixed numbers in your equation. To define a constant, we can do this:

\begin{lstlisting}[language=Python]
a = tf.constant(1, name='a_var')
b = tf.constant(2, name='b_bar')
\end{lstlisting}
Aside from the value 1, we can also provide a name such as \lstinline{a_var} for the tensor which is separate from the Python variable name \lstinline{a}. It's optional but will be helpful in later operations and troubleshooting.

After defining, if we print variable a, we'll have:

\begin{lstlisting}[language=Python]
<tf.Tensor 'a_var:0' shape=() dtype=int32>
\end{lstlisting}
Variables are the model parameters to be optimized, for example, the weights and biases in your neural networks. Similarly, we can also define a variable and show its contents like this:

\begin{lstlisting}
c = tf.Variable(a + b)
c
\end{lstlisting}
and have this output:

\begin{lstlisting}
<tf.Variable 'Variable:0' shape=() dtype=int32_ref>
\end{lstlisting}
But it's important to note that all variables need to be initialized before use like this:

\begin{lstlisting}
init = tf.global_variables_initializer()
\end{lstlisting}
You might have noticed that the values of a and b, i.e., integers 1 and 2 are not showing up anywhere, why?

That's an important characteristic of TensorFlow --- ``lazy execution'', meaning things are defined first, but not run. It's only executed when we tell it to do, which is done through the running of a session! (Note that TensorFlow also has eager execution. Check here for more info)

\subsection{Session and Computational Graph}

Now let's define a session and run it:

\begin{lstlisting}
with tf.Session() as session:                    
    session.run(init)                            
    print(session.run(c))
\end{lstlisting}
Notice that within the session we run both the initialization of variables and the calculation of c. We defined c as the sum of a and b:

\begin{lstlisting}
c = tf.Variable(a + b)
\end{lstlisting}
This, in TensorFlow and Deep Learning speak, is the ``computational graph''. Sounds pretty sophisticated, right? But it's really just an expression of the calculation we want to carry out!

\subsection{Placeholder}

Another important tensor type is the placeholder. Its use case is to hold the place for data to be supplied. For example, we defined a computational graph, and we have lots of training data, we can then use placeholders to indicate we'll feed these in later.

Let's see an example. Say we have an equation like this:

\begin{equation}
y = ax^2+bbx+c
\end{equation}

Instead of one single x input, we have a vector of x's. So we can use a placeholder to define x:

\begin{lstlisting}
x = tf.placeholder(dtype=tf.float32)
\end{lstlisting}
We also need the coefficients. Let's use constants:

\begin{lstlisting}
a = tf.constant(1, dtype=tf.float32)
b = tf.constant(-20, dtype=tf.float32)
c = tf.constant(-100, dtype=tf.float32)
\end{lstlisting}
Now let's make the computational graph and provide the input values for x:

\begin{lstlisting}
y = a * (x ** 2) + b * x + c
x_feed = np.linspace(-10, 30, num=10)
\end{lstlisting}
And finally, we can run it:

\begin{lstlisting}
with tf.Session() as sess:
  results = sess.run(y, feed_dict={x: x_feed})
print(results)
\end{lstlisting}
which gives us:

\begin{lstlisting}
[ 200.         41.975304  -76.54321  -155.55554  -195.06174  -195.06174  -155.55554   -76.54324    41.97534   200.      ]
\end{lstlisting}
\subsection{Putting it All Together}

Now that we have the basics of TensorFlow, let's do a mini project to build a linear regression model, aka, a neural network :) (The code is adapted from the example given in TensorFlow's guide here)

Let's say we have a bunch of x, y value pairs, and we need to find the best fit line. First, since both x and y have values to be fed in the model, we'll define them as placeholders:

\begin{lstlisting}
x = tf.placeholder(dtype=tf.float32, shape=(None, 1))
y_true = tf.placeholder(dtype=tf.float32, shape=(None, 1))
\end{lstlisting}
The number of rows is defined as None to have the flexibility of feeding in any number of rows we want.

Next, we need to define a model. In this case here, our model has just one layer with one weight and one bias.

\begin{marginfigure}
\includegraphics{colabrepo03}
\end{marginfigure}


TensorFlow allows us to define neural network layers very easily:

\begin{lstlisting}
linear_model = tf.layers.Dense(
                   units=1, 
                   bias_initializer=tf.constant_initializer(1))
y_pred = linear_model(x)
\end{lstlisting}
The number of units is set to be one since we only have one node in the hidden layer.

Furthermore, we need to have a loss function and set up the optimization method. The loss function is basically a way to measure how bad our model is when measured using the training data, so of course, we want it to be minimized. We'll use the gradient descent algorithm to optimize this loss function (I'll explain gradient descent in a future post).

\begin{lstlisting}
optimizer = tf.train.GradientDescentOptimizer(0.01)
train = optimizer.minimize(loss)
\end{lstlisting}
Then we can initialize all the variables. In this case here, all our variables including weight and bias are part of the layer we defined above.

\begin{lstlisting}
init = tf.global_variables_initializer()
\end{lstlisting}
Lastly, we can supply the training data for the placeholders and start the training:

\begin{lstlisting}
x_values = np.array([[1], [2], [3], [4]])
y_values = np.array([[0], [-1], [-2], [-3]])

with tf.Session() as sess:
  sess.run(init)
  for i in range(1000):
    _, loss_value = sess.run((train, loss),
                             feed_dict={x: x_values, y_true: y_values})
\end{lstlisting}
and we can get the weights and make the predictions like so:

\begin{lstlisting}
weights = sess.run(linear_model.weights)
bias = sess.run(linear_model.bias)
preds = sess.run(y_pred, 
                 feed_dict={x: x_values})
\end{lstlisting}
which yields these predictions:

\begin{lstlisting}
[[-0.00847495]  [-1.0041066 ]  [-1.9997383 ]  [-2.99537   ]]
\end{lstlisting}
If you are curious like me, you can verify to confirm the model did make the predictions using its trained weight and bias by:

\begin{lstlisting}
w = weights[0].tolist()[0][0]
b = weights[1].tolist()[0]
x_values * w + b
\end{lstlisting}
which gives us exactly the same result!

\begin{lstlisting}
array([[-0.00847495],        [-1.00410664],        [-1.99973834],        [-2.99537003]])
\end{lstlisting}
Voila! A simple neural network built using TensorFlow in Google Colab! Hope you find this tutorial interesting and informative.

\subsection{Final thoughts}

Cloud computing is definitely the future of Deep Learning computing. Google Colab is clearly a future-ready product. It's hard to imagine people still wanting to spend time setting up Deep Learning environments when we can just fire up a notebook on the cloud and start building models!


 % google colab
% -*- TeX -*- -*- UK -*- -*- Soft -*-

\chapter{Generalisation, Underfitting and Overfitting}
\label{sec:UnderfittingandOverfitting}

\section{Overfitting and Underfitting in Machine Learning}

This section is based on a blog by MLK \cite{MLK2019}, some parts are copied verbatim. There are some other nice pages here as well, see \cite{MLK2018}.

Generalisation is the ability of  neural net to not only learn the specific data, but to be able to also handle previously unseen data correctly in a broader statistical sense. 

Given a data set, there must be some True Function that can generalise the input-to-output mapping in the best statistical sense, for both the learning data as well as previously unseen data.  The objective is to train the net to find the True Function, but it can be quite hard to find this True Function.
There are two measures that can help describe the goodness of fit for the trained neural net.

\newthought{Bias} is the error that is introduced by the model's prediction and the actual data: Bias = Predicted – Actual.  Bias is not to be confused with offset, but rather as a generalised error against the training set: how well does the net fit the training set?
\begin{itemize}
\item Low Bias means, the model has created a function that has understood the relationship between input and output data.
\item  High Bias means, the model has created a function that fails to understand the relationship between input and output data.
\end{itemize}

\newthought{Variance} of a machine learning model is the amount by which its performance vary with \textit{different and previously unseen} data set.
\begin{itemize}
\item  Low variance means, the performance of machine learning model does not vary much with different data set
\item  High variance means, the performance of the machine learning model varies considerably with different data set.
\end{itemize}
A good fit model should be generalized enough to work with any unseen data (low variance) and at the same time should produce low prediction error (low bias).

\newthought{Overfitting} is when  the neural net model creates a function which correctly maps  the input data set to the required output data set, with little or no error.  The model so carefully adjusts to the training set that the follows the minutest of details and lose the bigger picture: it does not generalise well against new data.  An overfitted model has low bias, but high variance.

Overfitting can be detected by poor fit performance on the test data set (new data not in training set).  The model is so narrowly trained on the training set that it is unable to generalise well to the test data set.

Avoiding Overfitting by 
\begin{itemize}
\item Increase the data in your training set. Limiting yourself with a very small data set can cause your model to create a direct function rather than a generalized function.
\item Reduce the complexity of your machine learning model architecture. For example you can reduce the number of neurons/layers in neural network, reduce the value of K in KNN model, reduce number of estimators in Random Forest etc. Do remember that we should design the simplest machine learning model that can solve our problem.
\item Early stopping during training phase can prevent the model from overfitting with the training data itself in subsequent epochs.
\item In case of deep neural network you may use techniques of Dropouts where neurons are randomly switched off during training phase.
\item Applying L1 and L2 regularization techniques limits the models tendency to overfit. It is a broad topic which we may discuss in a separate post.
\end{itemize}


\newthought{Underfitting} is when the neural net model fails to build a model that sufficiently well generalises the True Function:  it is unable to capture the true relationship between the given input vectors and the required output vectors.
An underfitted model may or may not have low variance but it will have large bias.  

Underfitting can be detected by poor fit performance on the training data set: it fails to capture the essence of the True Function.

Avoiding Underfitting by
\begin{itemize}
\item Increase the data. If you limit the data during training phase you are not providing enough details to model to learn the relationship between data.

\item Increase the complexity of your machine learning model as it might help you to capture the underlying complex relation between data. For e.g increasing number of neurons/layers in neural network, increasing the value of K in KNN etc.

\item You might be using too few features and hence providing too much less information to your model, resulting in underfitting. So you may try to increase the number of feature into your model.

\item  There may be too much noise in the data that might be preventing your model to understand the correct nature of data. Try to do a proper preprocessing and noise removal from the data.

\item  You may be terminating the training epochs before model starts fitting the data. You may increase the number of epochs and see if the model shows more better accuracy.
\end{itemize}


 % underfitting and overfitting

%\part{Evolutionary Algorithms}
% -*- TeX -*- -*- UK -*- -*- Soft -*-


\part{Evolutionary Algorithms}

\chapter{Evolutionary Algorithms Overview}
\label{chap:EvolAlgoOverview}


\section{Introduction}

``In artificial intelligence, an \ac{EAlg} is a subset of evolutionary computation, a generic population-based metaheuristic optimization algorithm. An EA uses mechanisms inspired by biological evolution, such as reproduction, mutation, recombination, and selection. Candidate solutions to the optimization problem play the role of individuals in a population, and the fitness function determines the quality of the solutions. Evolution of the population then takes place after the repeated application of the above operators.

``Evolutionary algorithms often perform well approximating solutions to all types of problems because they ideally do not make any assumption about the underlying fitness landscape.  In most real applications of EAs, computational complexity is a prohibiting factor. In fact, this computational complexity is due to fitness function evaluation. Fitness approximation is one of the solutions to overcome this difficulty. However, seemingly simple EA can solve often complex problems; therefore, there may be no direct link between algorithm complexity and problem complexity.''\cite{WikipeadiaEvolutionaryAlgo2019}

\section{Overview}

The information in this section is taken from \cite{DevinSoni2018}.

Evolutionary algorithms are a heuristic-based approach to solving problems that cannot be easily solved in polynomial time, such as classically NP-Hard problems, and anything else that would take far too long to exhaustively process. When used on their own, they are typically applied to combinatorial problems; however, genetic algorithms are often used in tandem with other methods, acting as a quick way to find a somewhat optimal starting place for another algorithm to work off of.

The premise of an evolutionary algorithm (to be further known as an EA) is quite simple given that you are familiar with the process of natural selection. An EA contains four overall steps: initialization, selection, genetic operators, and termination. These steps each correspond, roughly, to a particular facet of natural selection, and provide easy ways to modularize implementations of this algorithm category. Simply put, in an EA, fitter members will survive and proliferate, while unfit members will die off and not contribute to the gene pool of further generations, much like in natural selection.

\begin{marginfigure}
\includegraphics{GA-overview}
\end{marginfigure}

We generally define the problem as such: we wish to find the best combination of elements that maximizes some fitness function, and we will accept a final solution once we have either ran the algorithm for some maximum number of iterations, or we have reached some fitness threshold. This scenario is clearly not the only way to use an EA, but it does encompass many common applications in the discrete case

\subsection{Initialization}
In order to begin our algorithm, we must first create an initial population of solutions. The population will contain an arbitrary number of possible solutions to the problem, oftentimes called members. It will often be created randomly (within the constraints of the problem) or, if some prior knowledge of the task is known, roughly centered around what is believed to be ideal. It is important that the population encompasses a wide range of solutions, because it essentially represents a gene pool; ergo, if we wish to explore many different possibilities over the course of the algorithm, we should aim to have many different genes present.

\subsection{Selection}
Once a population is created, members of the population must now be evaluated according to a fitness function. A fitness function is a function that takes in the characteristics of a member, and outputs a numerical representation of how viable of a solution it is. Creating the fitness function can often be very difficult, and it is important to find a good function that accurately represents the data; it is very problem-specific. Now, we calculate the fitness of all members, and select a portion of the top-scoring members.

\subsection{Multiple objective functions}
EAs can also be extended to use multiple fitness functions. This complicates the process somewhat, because instead of being able to identify a single optimal point, we instead end up with a set of optimal points when using multiple fitness functions. The set of optimal solutions is called the Pareto frontier, and contains elements that are equally optimal in the sense that no solution dominates any other solution in the frontier. A decider is then used to narrow the set down a single solution, based on the context of the problem or some other metric.


\begin{marginfigure}
\includegraphics{GA-pareto-frontier}
\end{marginfigure}


\subsection{Genetic Operators}
This step really includes two sub-steps: crossover and mutation. After selecting the top members (typically top 2, but this number can vary), these members are now used to create the next generation in the algorithm. Using the characteristics of the selected parents, new children are created that are a mixture of the parents’ qualities. Doing this can often be difficult depending on the type of data, but typically in combinatorial problems, it is possible to mix combinations and output valid combinations from these inputs. Now, we must introduce new genetic material into the generation. If we do not do this crucial step, we will become stuck in local extrema very quickly, and will not obtain optimal results. This step is mutation, and we do this, quite simply, by changing a small portion of the children such that they no longer perfectly mirror subsets of the parents’ genes. Mutation typically occurs probabilistically, in that the chance of a child receiving a mutation as well as the severity of the mutation are governed by a probability distribution.

\subsection{Termination}
Eventually, the algorithm must end. There are two cases in which this usually occurs: either the algorithm has reached some maximum runtime, or the algorithm has reached some threshold of performance. At this point a final solution is selected and returned.


 % Introduction
% -*- TeX -*- -*- UK -*- -*- Soft -*-

\chapter{DEAP}
\label{chap:DEAP}


\TBC{TBC}

 % DEAP
% -*- TeX -*- -*- UK -*- -*- Soft -*-

\chapter{Examples}
\label{chap:EAExamples}


\TBC{TBC}

 % Examples

%\part{Reference}
% -*- TeX -*- -*- UK -*- -*- Soft -*-

\part{Reference}

%\chapter{References}

\begin{fullwidth}

\bibliography{bib/AIMLDL} % data in 
\label{sec:references}
\end{fullwidth}
 %[[Done]] bibliography   

\end{document}
